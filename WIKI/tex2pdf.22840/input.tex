\documentclass[]{article}
\usepackage{lmodern}
\usepackage{amssymb,amsmath}
\usepackage{ifxetex,ifluatex}
\usepackage{fixltx2e} % provides \textsubscript
\ifnum 0\ifxetex 1\fi\ifluatex 1\fi=0 % if pdftex
  \usepackage[T1]{fontenc}
  \usepackage[utf8]{inputenc}
\else % if luatex or xelatex
  \ifxetex
    \usepackage{mathspec}
  \else
    \usepackage{fontspec}
  \fi
  \defaultfontfeatures{Ligatures=TeX,Scale=MatchLowercase}
\fi
% use upquote if available, for straight quotes in verbatim environments
\IfFileExists{upquote.sty}{\usepackage{upquote}}{}
% use microtype if available
\IfFileExists{microtype.sty}{%
\usepackage[]{microtype}
\UseMicrotypeSet[protrusion]{basicmath} % disable protrusion for tt fonts
}{}
\PassOptionsToPackage{hyphens}{url} % url is loaded by hyperref
\usepackage[unicode=true]{hyperref}
\hypersetup{
            pdfborder={0 0 0},
            breaklinks=true}
\urlstyle{same}  % don't use monospace font for urls
\usepackage{longtable,booktabs}
% Fix footnotes in tables (requires footnote package)
\IfFileExists{footnote.sty}{\usepackage{footnote}\makesavenoteenv{long table}}{}
\IfFileExists{parskip.sty}{%
\usepackage{parskip}
}{% else
\setlength{\parindent}{0pt}
\setlength{\parskip}{6pt plus 2pt minus 1pt}
}
\setlength{\emergencystretch}{3em}  % prevent overfull lines
\providecommand{\tightlist}{%
  \setlength{\itemsep}{0pt}\setlength{\parskip}{0pt}}
\setcounter{secnumdepth}{0}
% Redefines (sub)paragraphs to behave more like sections
\ifx\paragraph\undefined\else
\let\oldparagraph\paragraph
\renewcommand{\paragraph}[1]{\oldparagraph{#1}\mbox{}}
\fi
\ifx\subparagraph\undefined\else
\let\oldsubparagraph\subparagraph
\renewcommand{\subparagraph}[1]{\oldsubparagraph{#1}\mbox{}}
\fi

% set default figure placement to htbp
\makeatletter
\def\fps@figure{htbp}
\makeatother


\date{}

\begin{document}

\section{ChatScript System Variables and Engine-defined
Concepts}\label{chatscript-system-variables-and-engine-defined-concepts}

Copyright Bruce Wilcox, gowilcox@gmail.com www.brilligunderstanding.com
Revision 10/16/2022 cs12.3

\begin{itemize}
\tightlist
\item
  \href{ChatScript-System-Variables-and-Engine-defined-Concepts.md\#engine-defined-concepts}{Engine-defined
  Concepts}
\item
  \href{ChatScript-System-Variables-and-Engine-defined-Concepts.md\#system-variables}{System
  Variables}
\item
  \href{ChatScript-System-Variables-and-Engine-defined-Concepts.md\#control-over-input}{Control
  over Input}
\item
  \href{ChatScript-System-Variables-and-Engine-defined-Concepts.md\#interchange-variables}{Interchange
  Variables}
\end{itemize}

\section{Engine-defined concepts}\label{engine-defined-concepts}

In addition to concepts defined in script files, the system
automatically defines a bunch of dictionary-based sets as well as
dynamically computed concept members.

\begin{longtable}[]{@{}ll@{}}
\toprule
\begin{minipage}[b]{0.22\columnwidth}\raggedright\strut
set\strut
\end{minipage} & \begin{minipage}[b]{0.41\columnwidth}\raggedright\strut
description\strut
\end{minipage}\tabularnewline
\midrule
\endhead
\begin{minipage}[t]{0.22\columnwidth}\raggedright\strut
\texttt{\textasciitilde{}web\_url}\strut
\end{minipage} & \begin{minipage}[t]{0.41\columnwidth}\raggedright\strut
word is a web url\strut
\end{minipage}\tabularnewline
\begin{minipage}[t]{0.22\columnwidth}\raggedright\strut
\texttt{\textasciitilde{}email\_url}\strut
\end{minipage} & \begin{minipage}[t]{0.41\columnwidth}\raggedright\strut
word is an email address\strut
\end{minipage}\tabularnewline
\begin{minipage}[t]{0.22\columnwidth}\raggedright\strut
\texttt{\textasciitilde{}kindergarten}\strut
\end{minipage} & \begin{minipage}[t]{0.41\columnwidth}\raggedright\strut
word learned early in life\strut
\end{minipage}\tabularnewline
\begin{minipage}[t]{0.22\columnwidth}\raggedright\strut
\texttt{\textasciitilde{}grade1\_2}\strut
\end{minipage} & \begin{minipage}[t]{0.41\columnwidth}\raggedright\strut
word learned in these grades\strut
\end{minipage}\tabularnewline
\begin{minipage}[t]{0.22\columnwidth}\raggedright\strut
\texttt{\textasciitilde{}grade3\_4}\strut
\end{minipage} & \begin{minipage}[t]{0.41\columnwidth}\raggedright\strut
word learned in these grades\strut
\end{minipage}\tabularnewline
\begin{minipage}[t]{0.22\columnwidth}\raggedright\strut
\texttt{\textasciitilde{}grade\_5-6}\strut
\end{minipage} & \begin{minipage}[t]{0.41\columnwidth}\raggedright\strut
word learned in these grades. Unmarked words are learned even
later\strut
\end{minipage}\tabularnewline
\begin{minipage}[t]{0.22\columnwidth}\raggedright\strut
\texttt{\textasciitilde{}utf8}\strut
\end{minipage} & \begin{minipage}[t]{0.41\columnwidth}\raggedright\strut
word has nonascii characters\strut
\end{minipage}\tabularnewline
\begin{minipage}[t]{0.22\columnwidth}\raggedright\strut
\texttt{\textasciitilde{}daynumber}\strut
\end{minipage} & \begin{minipage}[t]{0.41\columnwidth}\raggedright\strut
word could be a number of a day in a month\strut
\end{minipage}\tabularnewline
\begin{minipage}[t]{0.22\columnwidth}\raggedright\strut
\texttt{\textasciitilde{}yearnumber}\strut
\end{minipage} & \begin{minipage}[t]{0.41\columnwidth}\raggedright\strut
word could be the number of a recent year\strut
\end{minipage}\tabularnewline
\begin{minipage}[t]{0.22\columnwidth}\raggedright\strut
\texttt{\textasciitilde{}dateinfo}\strut
\end{minipage} & \begin{minipage}[t]{0.41\columnwidth}\raggedright\strut
phrase is month day year of some kind\strut
\end{minipage}\tabularnewline
\begin{minipage}[t]{0.22\columnwidth}\raggedright\strut
\texttt{\textasciitilde{}kelvin}\strut
\end{minipage} & \begin{minipage}[t]{0.41\columnwidth}\raggedright\strut
temperature marker\strut
\end{minipage}\tabularnewline
\begin{minipage}[t]{0.22\columnwidth}\raggedright\strut
\texttt{\textasciitilde{}celcius}\strut
\end{minipage} & \begin{minipage}[t]{0.41\columnwidth}\raggedright\strut
temperature marker\strut
\end{minipage}\tabularnewline
\begin{minipage}[t]{0.22\columnwidth}\raggedright\strut
\texttt{\textasciitilde{}fahrenheit}\strut
\end{minipage} & \begin{minipage}[t]{0.41\columnwidth}\raggedright\strut
temperature marker\strut
\end{minipage}\tabularnewline
\begin{minipage}[t]{0.22\columnwidth}\raggedright\strut
\texttt{\textasciitilde{}twitter\_name}\strut
\end{minipage} & \begin{minipage}[t]{0.41\columnwidth}\raggedright\strut
twitter user name\strut
\end{minipage}\tabularnewline
\begin{minipage}[t]{0.22\columnwidth}\raggedright\strut
\texttt{\textasciitilde{}hashtag\_label}\strut
\end{minipage} & \begin{minipage}[t]{0.41\columnwidth}\raggedright\strut
twitter topic reference\strut
\end{minipage}\tabularnewline
\bottomrule
\end{longtable}

\subsection{\texorpdfstring{Interjections, ``discourse acts'', and
concept
sets}{Interjections, discourse acts, and concept sets}}\label{interjections-discourse-acts-and-concept-sets}

Some words and phrases have interpretations based on whether they are at
sentence start or not. E.g., \emph{good day, mate} and \emph{It is a
good day} are different for \emph{good day}.

Likewise \emph{sure} and \emph{I am sure} are different.

Words that have a different meaning at the start of a sentence are
commonly called interjections.

In ChatScript these are defined by the
\texttt{livedata/interjections.txt} file. In addition, the file augments
this concept with ``discourse acts'', phrases that are like an
interjection. All interjections and discourse acts map to concept sets,
which come thru as the user input instead of what they wrote.

For example \emph{yes} and \emph{sure} and \emph{of course} are all
treated as meaning the discourse act of agreement in the interjections
file. So you don't see \emph{yes, I will go} coming out of the engine.

The interjections file will remap that to the sentence
\texttt{\textasciitilde{}yes}, breaking off that into its own sentence,
followed by \emph{I will go} as a new sentence.

These generic interjections (which are open to author control via
interjections.txt) are:

\begin{longtable}[]{@{}ll@{}}
\toprule
interjection & description\tabularnewline
\midrule
\endhead
\texttt{\textasciitilde{}yes} &\tabularnewline
\texttt{\textasciitilde{}no} &\tabularnewline
\texttt{\textasciitilde{}emomaybe} &\tabularnewline
\texttt{\textasciitilde{}emohello} &\tabularnewline
\texttt{\textasciitilde{}emogoodbye} &\tabularnewline
\texttt{\textasciitilde{}emohowzit} &\tabularnewline
\texttt{\textasciitilde{}emothanks} &\tabularnewline
\texttt{\textasciitilde{}emolaugh} &\tabularnewline
\texttt{\textasciitilde{}emohappy} &\tabularnewline
\texttt{\textasciitilde{}emosad} &\tabularnewline
\texttt{\textasciitilde{}emosurprise} &\tabularnewline
\texttt{\textasciitilde{}emomisunderstand} &\tabularnewline
\texttt{\textasciitilde{}emoskeptic} &\tabularnewline
\texttt{\textasciitilde{}emoignorance} &\tabularnewline
\texttt{\textasciitilde{}emobeg} &\tabularnewline
\texttt{\textasciitilde{}emobored} &\tabularnewline
\texttt{\textasciitilde{}emopain} &\tabularnewline
\texttt{\textasciitilde{}emoangry} &\tabularnewline
\texttt{\textasciitilde{}emocurse} &\tabularnewline
\texttt{\textasciitilde{}emodisgust} &\tabularnewline
\texttt{\textasciitilde{}emoprotest} &\tabularnewline
\texttt{\textasciitilde{}emoapology} &\tabularnewline
\texttt{\textasciitilde{}emomutual} &\tabularnewline
\bottomrule
\end{longtable}

Because all interjections at the start of a sentence are broken off into
their own sentence, this kind of pattern does not work:

\begin{verbatim}
u: (~yes _*)
\end{verbatim}

You cannot capture the rest of the sentence here, because it will be
part of the next sentence instead. This means interjections act somewhat
differently from other concepts.

If you use a word in a pattern which may get remapped on input, the
script compiler will issue a warning. Likely you should use the remapped
name instead.

The following concepts are triggered by exactly repeating either the
chatbot or oneself (to a repeat count of how often repeated). Repeats
are within a recency window of about 20 volleys.

\begin{longtable}[]{@{}ll@{}}
\toprule
concept & description\tabularnewline
\midrule
\endhead
\texttt{\textasciitilde{}repeatme} &\tabularnewline
\texttt{\textasciitilde{}repeatinput1} &\tabularnewline
\texttt{\textasciitilde{}repeatinput2} &\tabularnewline
\texttt{\textasciitilde{}repeatinput3} &\tabularnewline
\texttt{\textasciitilde{}repeatinput4} &\tabularnewline
\texttt{\textasciitilde{}repeatinput5} &\tabularnewline
\texttt{\textasciitilde{}repeatinput6} &\tabularnewline
\bottomrule
\end{longtable}

\subsection{POS (Part of Speech) Tags}\label{pos-part-of-speech-tags}

Words will have pos-tags attached, specififying both generic and
specific tag attributes, eg., \texttt{\textasciitilde{}noun},
\texttt{\textasciitilde{}noun\_singular}.

\subsubsection{Generic Specifics}\label{generic-specifics}

\begin{longtable}[]{@{}ll@{}}
\toprule
nouns & description\tabularnewline
\midrule
\endhead
\texttt{\textasciitilde{}noun} &\tabularnewline
\texttt{\textasciitilde{}noun\_singular} &\tabularnewline
\texttt{\textasciitilde{}noun\_plural} &\tabularnewline
\texttt{\textasciitilde{}noun\_proper\_singular} &\tabularnewline
\texttt{\textasciitilde{}noun\_proper\_plural} &\tabularnewline
\texttt{\textasciitilde{}noun\_gerund} &\tabularnewline
\texttt{\textasciitilde{}noun\_number} &\tabularnewline
\texttt{\textasciitilde{}noun\_infinitive} &\tabularnewline
\texttt{\textasciitilde{}noun\_omitted\_adjective} &\tabularnewline
\bottomrule
\end{longtable}

\begin{longtable}[]{@{}ll@{}}
\toprule
verbs & description\tabularnewline
\midrule
\endhead
\texttt{\textasciitilde{}verb} &\tabularnewline
\texttt{\textasciitilde{}verb\_present} &\tabularnewline
\texttt{\textasciitilde{}verb\_present\_3ps} &\tabularnewline
\texttt{\textasciitilde{}verb\_infinitive} &\tabularnewline
\texttt{\textasciitilde{}verb\_present\_participle} &\tabularnewline
\texttt{\textasciitilde{}verb\_past} &\tabularnewline
\texttt{\textasciitilde{}verb\_past\_participle} &\tabularnewline
\texttt{\textasciitilde{}aux\_verb} &\tabularnewline
\texttt{\textasciitilde{}aux\_verb\_present} &\tabularnewline
\texttt{\textasciitilde{}aux\_verb\_past} &\tabularnewline
\texttt{\textasciitilde{}aux\_verb\_future} &\tabularnewline
\texttt{\textasciitilde{}aux\_verb\_tenses} &\tabularnewline
\texttt{\textasciitilde{}aux\_be} &\tabularnewline
\texttt{\textasciitilde{}aux\_have} &\tabularnewline
\texttt{\textasciitilde{}aux\_do} &\tabularnewline
\bottomrule
\end{longtable}

Auxilliary verbs are segmented into normal ones and special ones. Normal
ones give their tense directly. Special ones give their root word. The
tense of the be/have/do verbs can be had via \texttt{\^{}properties()}
and testing for verb tenses

\begin{longtable}[]{@{}ll@{}}
\toprule
adjectives & description\tabularnewline
\midrule
\endhead
\texttt{\textasciitilde{}adjective} &\tabularnewline
\texttt{\textasciitilde{}adjective\_normal} &\tabularnewline
\texttt{\textasciitilde{}adjective\_number} &\tabularnewline
\texttt{\textasciitilde{}adjective\_noun} &\tabularnewline
\texttt{\textasciitilde{}adjective\_participle} &\tabularnewline
\bottomrule
\end{longtable}

\begin{longtable}[]{@{}ll@{}}
\toprule
adjectives in comparative form & description\tabularnewline
\midrule
\endhead
\texttt{\textasciitilde{}more\_form}\texttt{\textasciitilde{}most\_form}
&\tabularnewline
\texttt{\textasciitilde{}adverb} &\tabularnewline
\texttt{\textasciitilde{}adverb\_normal} &\tabularnewline
\bottomrule
\end{longtable}

\begin{longtable}[]{@{}ll@{}}
\toprule
adverbs in comparative form & description\tabularnewline
\midrule
\endhead
\texttt{\textasciitilde{}more\_form}\texttt{\textasciitilde{}most\_form}
&\tabularnewline
\texttt{\textasciitilde{}pronoun}\texttt{\textasciitilde{}pronoun\_subject}\texttt{\textasciitilde{}pronoun\_object}
&\tabularnewline
\texttt{\textasciitilde{}conjunction\_bits}\texttt{\textasciitilde{}conjunction\_coordinate}\texttt{\textasciitilde{}conjunction\_subordinate}
&\tabularnewline
\texttt{\textasciitilde{}determiner\_bits}\texttt{\textasciitilde{}determiner}\texttt{\textasciitilde{}pronoun\_possessive}\texttt{\textasciitilde{}predeterminer}
&\tabularnewline
\texttt{\textasciitilde{}possessive} & covers \emph{'} and \emph{'s} at
end of word\tabularnewline
\texttt{\textasciitilde{}to\_infinitive} & ``to'' when used before a
noun infinitive\tabularnewline
\texttt{\textasciitilde{}preposition}\texttt{\textasciitilde{}particle}
& free-floating preposition tied to idiomatic verb\tabularnewline
\texttt{\textasciitilde{}comma} &\tabularnewline
\texttt{\textasciitilde{}quote} & covers \emph{'} and \_``\_ when not
embedded in a word\tabularnewline
\texttt{\textasciitilde{}paren} & covers opening and closing
parens\tabularnewline
\texttt{\textasciitilde{}foreign\_word} & some unknown
word\tabularnewline
\texttt{\textasciitilde{}there\_existential} & the word there used
existentially\tabularnewline
\bottomrule
\end{longtable}

In addition to normal generic kinds of pos tags, words which are serving
a pos-tag role different from their putative word type are marked as
members of the major tag they act as part of. E.g,

\begin{longtable}[]{@{}ll@{}}
\toprule
\begin{minipage}[b]{0.41\columnwidth}\raggedright\strut
\strut
\end{minipage} & \begin{minipage}[b]{0.20\columnwidth}\raggedright\strut
description\strut
\end{minipage}\tabularnewline
\midrule
\endhead
\begin{minipage}[t]{0.41\columnwidth}\raggedright\strut
\texttt{\textasciitilde{}noun\_gerund}\strut
\end{minipage} & \begin{minipage}[t]{0.20\columnwidth}\raggedright\strut
verb used as a \textasciitilde{}noun\strut
\end{minipage}\tabularnewline
\begin{minipage}[t]{0.41\columnwidth}\raggedright\strut
\texttt{\textasciitilde{}noun\_infinitive}\strut
\end{minipage} & \begin{minipage}[t]{0.20\columnwidth}\raggedright\strut
verb used as a \textasciitilde{}noun\strut
\end{minipage}\tabularnewline
\begin{minipage}[t]{0.41\columnwidth}\raggedright\strut
\texttt{\textasciitilde{}noun\_omitted\_adjective}\strut
\end{minipage} & \begin{minipage}[t]{0.20\columnwidth}\raggedright\strut
an adjective used as a collective noun (eg \emph{the beautiful are
kind})\strut
\end{minipage}\tabularnewline
\begin{minipage}[t]{0.41\columnwidth}\raggedright\strut
\texttt{\textasciitilde{}adjectival\_noun}\strut
\end{minipage} & \begin{minipage}[t]{0.20\columnwidth}\raggedright\strut
noun used as adjective like bank ``bank teller''\strut
\end{minipage}\tabularnewline
\begin{minipage}[t]{0.41\columnwidth}\raggedright\strut
\texttt{\textasciitilde{}adjective\_participle}\strut
\end{minipage} & \begin{minipage}[t]{0.20\columnwidth}\raggedright\strut
verb participle used as an adjective\strut
\end{minipage}\tabularnewline
\bottomrule
\end{longtable}

For \texttt{\textasciitilde{}noun\_gerund} in \emph{I like swimming} the
verb gerund \emph{swimming} is treated as a noun (hence called
noun-gerund) but retains verb sense when matching keywords tagged with
part-of-speech (i.e., it would match \texttt{swim\textasciitilde{}v} as
well as \texttt{swim\textasciitilde{}n}).

Additionally, there is

\begin{longtable}[]{@{}ll@{}}
\toprule
\begin{minipage}[b]{0.30\columnwidth}\raggedright\strut
\strut
\end{minipage} & \begin{minipage}[b]{0.20\columnwidth}\raggedright\strut
description\strut
\end{minipage}\tabularnewline
\midrule
\endhead
\begin{minipage}[t]{0.30\columnwidth}\raggedright\strut
\texttt{\textasciitilde{}number}\strut
\end{minipage} & \begin{minipage}[t]{0.20\columnwidth}\raggedright\strut
is not a part of speech, but is comprise of
\texttt{\textasciitilde{}noun\_number} (a normal number value like
\emph{17} or \emph{seventeen})\strut
\end{minipage}\tabularnewline
\begin{minipage}[t]{0.30\columnwidth}\raggedright\strut
\texttt{\textasciitilde{}adjective\_number}\strut
\end{minipage} & \begin{minipage}[t]{0.20\columnwidth}\raggedright\strut
also a normal numeral value and also
\texttt{\textasciitilde{}placenumber}) like \emph{first}.\strut
\end{minipage}\tabularnewline
\begin{minipage}[t]{0.30\columnwidth}\raggedright\strut
\texttt{\textasciitilde{}integer}\strut
\end{minipage} & \begin{minipage}[t]{0.20\columnwidth}\raggedright\strut
\strut
\end{minipage}\tabularnewline
\begin{minipage}[t]{0.30\columnwidth}\raggedright\strut
\texttt{\textasciitilde{}float}\strut
\end{minipage} & \begin{minipage}[t]{0.20\columnwidth}\raggedright\strut
\strut
\end{minipage}\tabularnewline
\begin{minipage}[t]{0.30\columnwidth}\raggedright\strut
\texttt{\textasciitilde{}positiveinteger}\strut
\end{minipage} & \begin{minipage}[t]{0.20\columnwidth}\raggedright\strut
\strut
\end{minipage}\tabularnewline
\begin{minipage}[t]{0.30\columnwidth}\raggedright\strut
\texttt{\textasciitilde{}negativeinteger}\strut
\end{minipage} & \begin{minipage}[t]{0.20\columnwidth}\raggedright\strut
\strut
\end{minipage}\tabularnewline
\begin{minipage}[t]{0.30\columnwidth}\raggedright\strut
\texttt{\textasciitilde{}modelnumber}\strut
\end{minipage} & \begin{minipage}[t]{0.20\columnwidth}\raggedright\strut
not a true number, but a word with both alpha and numeric\strut
\end{minipage}\tabularnewline
\begin{minipage}[t]{0.30\columnwidth}\raggedright\strut
\texttt{\textasciitilde{}filename}\strut
\end{minipage} & \begin{minipage}[t]{0.20\columnwidth}\raggedright\strut
looks like a filename with extension\strut
\end{minipage}\tabularnewline
\bottomrule
\end{longtable}

To can be a preposition or it can be special. When used in the
infinitive phrase To go, it is marked
\texttt{\textasciitilde{}to\_infinitive} and is followed by
\texttt{\textasciitilde{}noun\_infinitive}.

\begin{longtable}[]{@{}ll@{}}
\toprule
\begin{minipage}[b]{0.30\columnwidth}\raggedright\strut
\strut
\end{minipage} & \begin{minipage}[b]{0.20\columnwidth}\raggedright\strut
description\strut
\end{minipage}\tabularnewline
\midrule
\endhead
\begin{minipage}[t]{0.30\columnwidth}\raggedright\strut
\texttt{\textasciitilde{}verb\_infinitive}\strut
\end{minipage} & \begin{minipage}[t]{0.20\columnwidth}\raggedright\strut
refers to a match on the infinitive form of the verb (\emph{I hear John
sing} or \emph{I will sing}).\strut
\end{minipage}\tabularnewline
\begin{minipage}[t]{0.30\columnwidth}\raggedright\strut
\texttt{\textasciitilde{}There\_existential}\strut
\end{minipage} & \begin{minipage}[t]{0.20\columnwidth}\raggedright\strut
refers to the use of where not involving location, meaning the existence
of, as in There is no future.\strut
\end{minipage}\tabularnewline
\begin{minipage}[t]{0.30\columnwidth}\raggedright\strut
\texttt{\textasciitilde{}Particle}\strut
\end{minipage} & \begin{minipage}[t]{0.20\columnwidth}\raggedright\strut
refers to a preposition piece of a compound verb idiom which allows
being separated from the verb. If you say \emph{I will call off the
meeting}, call\_off is the composite verb and is a single token. But if
you split it as in \emph{I will call the meeting off}, then there are
two tokens. The original form of the verb will be call and the canonical
form of the verb will be call\_off, while the free-standing off will be
labeled \texttt{\textasciitilde{}particle}.\strut
\end{minipage}\tabularnewline
\begin{minipage}[t]{0.30\columnwidth}\raggedright\strut
\texttt{\textasciitilde{}verb\_present}\strut
\end{minipage} & \begin{minipage}[t]{0.20\columnwidth}\raggedright\strut
will be used for normal present verbs not in third person singular like
\emph{I walk} and\strut
\end{minipage}\tabularnewline
\begin{minipage}[t]{0.30\columnwidth}\raggedright\strut
\texttt{\textasciitilde{}verb\_present\_3ps}\strut
\end{minipage} & \begin{minipage}[t]{0.20\columnwidth}\raggedright\strut
will be used for things like \emph{he walks}\strut
\end{minipage}\tabularnewline
\begin{minipage}[t]{0.30\columnwidth}\raggedright\strut
\texttt{\textasciitilde{}possesive}\strut
\end{minipage} & \begin{minipage}[t]{0.20\columnwidth}\raggedright\strut
refers to \emph{`s} and \emph{`} that indicate possession, while
possessive pronouns get their own labeling
\texttt{\textasciitilde{}pronoun\_possessive}.\strut
\end{minipage}\tabularnewline
\begin{minipage}[t]{0.30\columnwidth}\raggedright\strut
\texttt{\textasciitilde{}pronoun\_subject}\strut
\end{minipage} & \begin{minipage}[t]{0.20\columnwidth}\raggedright\strut
is a pronoun used as a subject (like \emph{he})\strut
\end{minipage}\tabularnewline
\begin{minipage}[t]{0.30\columnwidth}\raggedright\strut
\texttt{\textasciitilde{}pronoun\_object}\strut
\end{minipage} & \begin{minipage}[t]{0.20\columnwidth}\raggedright\strut
refers to objective form like \emph{him}\strut
\end{minipage}\tabularnewline
\bottomrule
\end{longtable}

Individual words serve roles in the parse of a sentence, which are
retrievable. These include:

\begin{longtable}[]{@{}ll@{}}
\toprule
\begin{minipage}[b]{0.30\columnwidth}\raggedright\strut
\strut
\end{minipage} & \begin{minipage}[b]{0.20\columnwidth}\raggedright\strut
description\strut
\end{minipage}\tabularnewline
\midrule
\endhead
\begin{minipage}[t]{0.30\columnwidth}\raggedright\strut
\texttt{\textasciitilde{}mainsubject}\strut
\end{minipage} & \begin{minipage}[t]{0.20\columnwidth}\raggedright\strut
\strut
\end{minipage}\tabularnewline
\begin{minipage}[t]{0.30\columnwidth}\raggedright\strut
\texttt{\textasciitilde{}mainverb}\strut
\end{minipage} & \begin{minipage}[t]{0.20\columnwidth}\raggedright\strut
\strut
\end{minipage}\tabularnewline
\begin{minipage}[t]{0.30\columnwidth}\raggedright\strut
\texttt{\textasciitilde{}mainindirect}\strut
\end{minipage} & \begin{minipage}[t]{0.20\columnwidth}\raggedright\strut
\strut
\end{minipage}\tabularnewline
\begin{minipage}[t]{0.30\columnwidth}\raggedright\strut
\texttt{\textasciitilde{}maindirect}\strut
\end{minipage} & \begin{minipage}[t]{0.20\columnwidth}\raggedright\strut
\strut
\end{minipage}\tabularnewline
\begin{minipage}[t]{0.30\columnwidth}\raggedright\strut
\texttt{\textasciitilde{}subject2}\strut
\end{minipage} & \begin{minipage}[t]{0.20\columnwidth}\raggedright\strut
\strut
\end{minipage}\tabularnewline
\begin{minipage}[t]{0.30\columnwidth}\raggedright\strut
\texttt{\textasciitilde{}verb2}\strut
\end{minipage} & \begin{minipage}[t]{0.20\columnwidth}\raggedright\strut
\strut
\end{minipage}\tabularnewline
\begin{minipage}[t]{0.30\columnwidth}\raggedright\strut
\texttt{\textasciitilde{}indirectobject2}\strut
\end{minipage} & \begin{minipage}[t]{0.20\columnwidth}\raggedright\strut
\strut
\end{minipage}\tabularnewline
\begin{minipage}[t]{0.30\columnwidth}\raggedright\strut
\texttt{\textasciitilde{}object2}\strut
\end{minipage} & \begin{minipage}[t]{0.20\columnwidth}\raggedright\strut
\strut
\end{minipage}\tabularnewline
\begin{minipage}[t]{0.30\columnwidth}\raggedright\strut
\texttt{\textasciitilde{}subject\_complement}\strut
\end{minipage} & \begin{minipage}[t]{0.20\columnwidth}\raggedright\strut
adjective object of sentence involving linking verb\strut
\end{minipage}\tabularnewline
\begin{minipage}[t]{0.30\columnwidth}\raggedright\strut
\texttt{\textasciitilde{}object\_complement}\strut
\end{minipage} & \begin{minipage}[t]{0.20\columnwidth}\raggedright\strut
2ndary noun or infinitive verb filling modifying mainobject or
object2\strut
\end{minipage}\tabularnewline
\begin{minipage}[t]{0.30\columnwidth}\raggedright\strut
\texttt{\textasciitilde{}conjunct\_noun}\texttt{\textasciitilde{}conjunct\_verb}\texttt{\textasciitilde{}conjunct\_adjective}\texttt{\textasciitilde{}conjunct\_adverb}\texttt{\textasciitilde{}conjunct\_phrase}\texttt{\textasciitilde{}conjunct\_clause}\texttt{\textasciitilde{}conjunct\_sentence}\strut
\end{minipage} & \begin{minipage}[t]{0.20\columnwidth}\raggedright\strut
\strut
\end{minipage}\tabularnewline
\begin{minipage}[t]{0.30\columnwidth}\raggedright\strut
\texttt{\textasciitilde{}postnominalAdjective}\strut
\end{minipage} & \begin{minipage}[t]{0.20\columnwidth}\raggedright\strut
adjective occuring AFTER the noun it modified\strut
\end{minipage}\tabularnewline
\begin{minipage}[t]{0.30\columnwidth}\raggedright\strut
\texttt{\textasciitilde{}reflexive}\strut
\end{minipage} & \begin{minipage}[t]{0.20\columnwidth}\raggedright\strut
reflexive pronouns\strut
\end{minipage}\tabularnewline
\begin{minipage}[t]{0.30\columnwidth}\raggedright\strut
\texttt{\textasciitilde{}not}\strut
\end{minipage} & \begin{minipage}[t]{0.20\columnwidth}\raggedright\strut
\strut
\end{minipage}\tabularnewline
\begin{minipage}[t]{0.30\columnwidth}\raggedright\strut
\texttt{\textasciitilde{}address}\strut
\end{minipage} & \begin{minipage}[t]{0.20\columnwidth}\raggedright\strut
noun used as addressee of sentence\strut
\end{minipage}\tabularnewline
\begin{minipage}[t]{0.30\columnwidth}\raggedright\strut
\texttt{\textasciitilde{}appositive}\strut
\end{minipage} & \begin{minipage}[t]{0.20\columnwidth}\raggedright\strut
noun restating and modifying prior noun\strut
\end{minipage}\tabularnewline
\begin{minipage}[t]{0.30\columnwidth}\raggedright\strut
\texttt{\textasciitilde{}absolutephrase}\strut
\end{minipage} & \begin{minipage}[t]{0.20\columnwidth}\raggedright\strut
special phrase describing whole sentence\strut
\end{minipage}\tabularnewline
\begin{minipage}[t]{0.30\columnwidth}\raggedright\strut
\texttt{\textasciitilde{}omittedtimeprep}\strut
\end{minipage} & \begin{minipage}[t]{0.20\columnwidth}\raggedright\strut
modified time word used as phrase but lacking preposition (\emph{Next
tuesday I will go})\strut
\end{minipage}\tabularnewline
\begin{minipage}[t]{0.30\columnwidth}\raggedright\strut
\texttt{\textasciitilde{}phrase}\strut
\end{minipage} & \begin{minipage}[t]{0.20\columnwidth}\raggedright\strut
a prepositional phrase start (except\strut
\end{minipage}\tabularnewline
\begin{minipage}[t]{0.30\columnwidth}\raggedright\strut
\texttt{\textasciitilde{}clause}\strut
\end{minipage} & \begin{minipage}[t]{0.20\columnwidth}\raggedright\strut
a subordinate clause start\strut
\end{minipage}\tabularnewline
\begin{minipage}[t]{0.30\columnwidth}\raggedright\strut
\texttt{\textasciitilde{}verbal}\strut
\end{minipage} & \begin{minipage}[t]{0.20\columnwidth}\raggedright\strut
a verb phrase\strut
\end{minipage}\tabularnewline
\bottomrule
\end{longtable}

and special concepts: \textbar{} \texttt{\textasciitilde{}capacronym}
\textbar{} word is in all caps (and \&) and is likely an acronym
\textbar{} \texttt{\textasciitilde{}emoji} \textbar{} word starts and
end with : and represents an emoji

\subsection{Spanish}\label{spanish}

For Spanish (if you are in spanish language mode) there is
\textasciitilde{}spanish\_he, \textasciitilde{}spanish\_she,
\textasciitilde{}spanish\_singular, \textasciitilde{}spanish\_plural for
nouns and adjectives and determiner `the'. Pronouns will be marked with
\textasciitilde{}pronoun\_object\_singular or
\textasciitilde{}pronoun\_object\_plural or
\textasciitilde{}pronoun\_object\_you. Also
\textasciitilde{}pronoun\_indirectobject\_singular and
\textasciitilde{}pronoun\_indirectobject\_plural and
\textasciitilde{}pronoun\_indirectobject\_you. Also
\textasciitilde{}pronoun\_I and \textasciitilde{}pronoun\_you. And
simple future tense verbs will be marked
\textasciitilde{}spanish\_future.

\section{System Variables}\label{system-variables}

The system has some predefined variables which you can generally test
and use but not normally assign to. These all begin with \texttt{\%} .
Ones that are reasonable to set are written in bold underline. Boolean
values are always \texttt{1} or \texttt{null} on returns. \texttt{1} or
\texttt{0} if you are setting them.

\subsection{Date \& Time \& Numbers}\label{date-time-numbers}

\begin{longtable}[]{@{}ll@{}}
\toprule
\begin{minipage}[b]{0.12\columnwidth}\raggedright\strut
variable\strut
\end{minipage} & \begin{minipage}[b]{0.61\columnwidth}\raggedright\strut
description\strut
\end{minipage}\tabularnewline
\midrule
\endhead
\begin{minipage}[t]{0.12\columnwidth}\raggedright\strut
\texttt{\%date}\strut
\end{minipage} & \begin{minipage}[t]{0.61\columnwidth}\raggedright\strut
one or two digit day of the month\strut
\end{minipage}\tabularnewline
\begin{minipage}[t]{0.12\columnwidth}\raggedright\strut
\texttt{\%day}\strut
\end{minipage} & \begin{minipage}[t]{0.61\columnwidth}\raggedright\strut
Sunday, etc\strut
\end{minipage}\tabularnewline
\begin{minipage}[t]{0.12\columnwidth}\raggedright\strut
\texttt{\%daynumber}\strut
\end{minipage} & \begin{minipage}[t]{0.61\columnwidth}\raggedright\strut
1-7 where 1 = Sunday\strut
\end{minipage}\tabularnewline
\begin{minipage}[t]{0.12\columnwidth}\raggedright\strut
\texttt{\%fulltime}\strut
\end{minipage} & \begin{minipage}[t]{0.61\columnwidth}\raggedright\strut
seconds representing the current time and date (Unix epoch time)\strut
\end{minipage}\tabularnewline
\begin{minipage}[t]{0.12\columnwidth}\raggedright\strut
\texttt{\%fullmstime}\strut
\end{minipage} & \begin{minipage}[t]{0.61\columnwidth}\raggedright\strut
Numeric full time/date in milliseconds (Unix epoch time)\strut
\end{minipage}\tabularnewline
\begin{minipage}[t]{0.12\columnwidth}\raggedright\strut
\texttt{\%hour}\strut
\end{minipage} & \begin{minipage}[t]{0.61\columnwidth}\raggedright\strut
0-23\strut
\end{minipage}\tabularnewline
\begin{minipage}[t]{0.12\columnwidth}\raggedright\strut
\texttt{\%timenumbers}\strut
\end{minipage} & \begin{minipage}[t]{0.61\columnwidth}\raggedright\strut
completely consistent full time info in numbers that you can do
\texttt{\_0\ =\ \^{}burst(\%timenumbers)}to get \texttt{\_0} =seconds
(2digit) \texttt{\_1}=minutes (2digit) \texttt{\_2}=hours (2digit)
\texttt{\_3}=dayinweek(0-6 Sunday=0) \texttt{\_4}=dateinmonth (1-31)
\texttt{\_5}=month(0-11 January=0) \texttt{\_6}=year.You need to get it
simultaneously if you want to do accurate things with current time,
since retrieving \%hour \%minute separately allows time to change
between calls\strut
\end{minipage}\tabularnewline
\begin{minipage}[t]{0.12\columnwidth}\raggedright\strut
\texttt{\%leapyear}\strut
\end{minipage} & \begin{minipage}[t]{0.61\columnwidth}\raggedright\strut
boolean if current year is a leap year\strut
\end{minipage}\tabularnewline
\begin{minipage}[t]{0.12\columnwidth}\raggedright\strut
\texttt{\%daylightsavings}\strut
\end{minipage} & \begin{minipage}[t]{0.61\columnwidth}\raggedright\strut
boolean if current within daylight savings\strut
\end{minipage}\tabularnewline
\begin{minipage}[t]{0.12\columnwidth}\raggedright\strut
\texttt{\%minute}\strut
\end{minipage} & \begin{minipage}[t]{0.61\columnwidth}\raggedright\strut
0-59\strut
\end{minipage}\tabularnewline
\begin{minipage}[t]{0.12\columnwidth}\raggedright\strut
\texttt{\%month}\strut
\end{minipage} & \begin{minipage}[t]{0.61\columnwidth}\raggedright\strut
1-12 (January = 1)\strut
\end{minipage}\tabularnewline
\begin{minipage}[t]{0.12\columnwidth}\raggedright\strut
\texttt{\%monthname}\strut
\end{minipage} & \begin{minipage}[t]{0.61\columnwidth}\raggedright\strut
January, etc\strut
\end{minipage}\tabularnewline
\begin{minipage}[t]{0.12\columnwidth}\raggedright\strut
\texttt{\%second}\strut
\end{minipage} & \begin{minipage}[t]{0.61\columnwidth}\raggedright\strut
0-59\strut
\end{minipage}\tabularnewline
\begin{minipage}[t]{0.12\columnwidth}\raggedright\strut
\texttt{\%volleytime}\strut
\end{minipage} & \begin{minipage}[t]{0.61\columnwidth}\raggedright\strut
number of seconds of computation since volley input started\strut
\end{minipage}\tabularnewline
\begin{minipage}[t]{0.12\columnwidth}\raggedright\strut
\texttt{\%time}\strut
\end{minipage} & \begin{minipage}[t]{0.61\columnwidth}\raggedright\strut
hh:mm in military 24-hour time\strut
\end{minipage}\tabularnewline
\begin{minipage}[t]{0.12\columnwidth}\raggedright\strut
\texttt{\%zulutime}\strut
\end{minipage} & \begin{minipage}[t]{0.61\columnwidth}\raggedright\strut
2016-07-27T11:38:35.253Z\strut
\end{minipage}\tabularnewline
\begin{minipage}[t]{0.12\columnwidth}\raggedright\strut
\texttt{\%week}\strut
\end{minipage} & \begin{minipage}[t]{0.61\columnwidth}\raggedright\strut
1-5 (week of the month)\strut
\end{minipage}\tabularnewline
\begin{minipage}[t]{0.12\columnwidth}\raggedright\strut
\texttt{\%year}\strut
\end{minipage} & \begin{minipage}[t]{0.61\columnwidth}\raggedright\strut
e.g., 2011\strut
\end{minipage}\tabularnewline
\begin{minipage}[t]{0.12\columnwidth}\raggedright\strut
\texttt{\%rand}\strut
\end{minipage} & \begin{minipage}[t]{0.61\columnwidth}\raggedright\strut
get a random number from 1 to 100 inclusive\strut
\end{minipage}\tabularnewline
\bottomrule
\end{longtable}

Time and date information are normally local, relative to the system
clock of the machine CS is running on. See \$cs\_utcoffset for adjusting
time based on relationship to utc (e.g your server is in Virginia and
you are in Colorado).

\%rand is only pseudo-random. A specific username is assigned a seed
based on their name. Thereafter the seed evolves by the dialog but it is
repeatable when the same user starts over again. If you want truly
random, use \%fullmstime \% \$howmany to get range 0 .. \$howmany-1

\subsection{User Input}\label{user-input}

\begin{longtable}[]{@{}ll@{}}
\toprule
\begin{minipage}[b]{0.26\columnwidth}\raggedright\strut
variable\strut
\end{minipage} & \begin{minipage}[b]{0.10\columnwidth}\raggedright\strut
description\strut
\end{minipage}\tabularnewline
\midrule
\endhead
\begin{minipage}[t]{0.26\columnwidth}\raggedright\strut
\texttt{\%bot}\strut
\end{minipage} & \begin{minipage}[t]{0.10\columnwidth}\raggedright\strut
current bot responding\strut
\end{minipage}\tabularnewline
\begin{minipage}[t]{0.26\columnwidth}\raggedright\strut
\texttt{\%revisedinput}\strut
\end{minipage} & \begin{minipage}[t]{0.10\columnwidth}\raggedright\strut
Boolean is current input from \texttt{\^{}input} not direct from
user\strut
\end{minipage}\tabularnewline
\begin{minipage}[t]{0.26\columnwidth}\raggedright\strut
\texttt{\%command}\strut
\end{minipage} & \begin{minipage}[t]{0.10\columnwidth}\raggedright\strut
Boolean was the user input a command\strut
\end{minipage}\tabularnewline
\begin{minipage}[t]{0.26\columnwidth}\raggedright\strut
\texttt{\%foreign}\strut
\end{minipage} & \begin{minipage}[t]{0.10\columnwidth}\raggedright\strut
Boolean is bulk of the sentence composed of foreign words\strut
\end{minipage}\tabularnewline
\begin{minipage}[t]{0.26\columnwidth}\raggedright\strut
\texttt{\%impliedyou}\strut
\end{minipage} & \begin{minipage}[t]{0.10\columnwidth}\raggedright\strut
Boolean was the user input having you as implied subject\strut
\end{minipage}\tabularnewline
\begin{minipage}[t]{0.26\columnwidth}\raggedright\strut
\texttt{\%impliedsubject}\strut
\end{minipage} & \begin{minipage}[t]{0.10\columnwidth}\raggedright\strut
Boolean was the user input having an implied subject (not you, usually
I)\strut
\end{minipage}\tabularnewline
\begin{minipage}[t]{0.26\columnwidth}\raggedright\strut
\texttt{\%input}\strut
\end{minipage} & \begin{minipage}[t]{0.10\columnwidth}\raggedright\strut
the count of the number of volleys this user has made ever\strut
\end{minipage}\tabularnewline
\begin{minipage}[t]{0.26\columnwidth}\raggedright\strut
\texttt{\%volley}\strut
\end{minipage} & \begin{minipage}[t]{0.10\columnwidth}\raggedright\strut
sae as \%input, the count of the number of volleys this user has made
ever\strut
\end{minipage}\tabularnewline
\begin{minipage}[t]{0.26\columnwidth}\raggedright\strut
\texttt{\%ip}\strut
\end{minipage} & \begin{minipage}[t]{0.10\columnwidth}\raggedright\strut
ip address supplied\strut
\end{minipage}\tabularnewline
\begin{minipage}[t]{0.26\columnwidth}\raggedright\strut
\texttt{\%myip}\strut
\end{minipage} & \begin{minipage}[t]{0.10\columnwidth}\raggedright\strut
ip address of cs server responding\strut
\end{minipage}\tabularnewline
\begin{minipage}[t]{0.26\columnwidth}\raggedright\strut
\texttt{\%language}\strut
\end{minipage} & \begin{minipage}[t]{0.10\columnwidth}\raggedright\strut
current dictionary language\strut
\end{minipage}\tabularnewline
\begin{minipage}[t]{0.26\columnwidth}\raggedright\strut
\texttt{\%length}\strut
\end{minipage} & \begin{minipage}[t]{0.10\columnwidth}\raggedright\strut
the length in tokens of the current sentence\strut
\end{minipage}\tabularnewline
\begin{minipage}[t]{0.26\columnwidth}\raggedright\strut
\texttt{\%more}\strut
\end{minipage} & \begin{minipage}[t]{0.10\columnwidth}\raggedright\strut
Boolean is there another sentence after this\strut
\end{minipage}\tabularnewline
\begin{minipage}[t]{0.26\columnwidth}\raggedright\strut
\texttt{\%morequestion}\strut
\end{minipage} & \begin{minipage}[t]{0.10\columnwidth}\raggedright\strut
Boolean is there a \texttt{?} or question word in the pending
sentences\strut
\end{minipage}\tabularnewline
\begin{minipage}[t]{0.26\columnwidth}\raggedright\strut
\texttt{\%originalinput}\strut
\end{minipage} & \begin{minipage}[t]{0.10\columnwidth}\raggedright\strut
all sentences user passed into volley, before adjusted in any way except
OOB data is stripped off\strut
\end{minipage}\tabularnewline
\begin{minipage}[t]{0.26\columnwidth}\raggedright\strut
\texttt{\%originalsentence}\strut
\end{minipage} & \begin{minipage}[t]{0.10\columnwidth}\raggedright\strut
the current sentence after tokenization but before any adjustments\strut
\end{minipage}\tabularnewline
\begin{minipage}[t]{0.26\columnwidth}\raggedright\strut
\texttt{\%parsed}\strut
\end{minipage} & \begin{minipage}[t]{0.10\columnwidth}\raggedright\strut
Boolean was current input parsed successfully\strut
\end{minipage}\tabularnewline
\begin{minipage}[t]{0.26\columnwidth}\raggedright\strut
\texttt{\%question}\strut
\end{minipage} & \begin{minipage}[t]{0.10\columnwidth}\raggedright\strut
Boolean was the user input a question - same as \texttt{?} in a
pattern\strut
\end{minipage}\tabularnewline
\begin{minipage}[t]{0.26\columnwidth}\raggedright\strut
\texttt{\%quotation}\strut
\end{minipage} & \begin{minipage}[t]{0.10\columnwidth}\raggedright\strut
Boolean is current input a quotation\strut
\end{minipage}\tabularnewline
\begin{minipage}[t]{0.26\columnwidth}\raggedright\strut
\texttt{\%sentence}\strut
\end{minipage} & \begin{minipage}[t]{0.10\columnwidth}\raggedright\strut
Boolean does it seem like a sentence (subject/verb or command)\strut
\end{minipage}\tabularnewline
\begin{minipage}[t]{0.26\columnwidth}\raggedright\strut
\texttt{\%tableinput}\strut
\end{minipage} & \begin{minipage}[t]{0.10\columnwidth}\raggedright\strut
current line being executed in a table expansion during script
compilation\strut
\end{minipage}\tabularnewline
\begin{minipage}[t]{0.26\columnwidth}\raggedright\strut
\texttt{\%tense}\strut
\end{minipage} & \begin{minipage}[t]{0.10\columnwidth}\raggedright\strut
past , present, or future simple tense (present perfect is a past
tense)\strut
\end{minipage}\tabularnewline
\begin{minipage}[t]{0.26\columnwidth}\raggedright\strut
\texttt{\%user}\strut
\end{minipage} & \begin{minipage}[t]{0.10\columnwidth}\raggedright\strut
user login name supplied\strut
\end{minipage}\tabularnewline
\begin{minipage}[t]{0.26\columnwidth}\raggedright\strut
\texttt{\%userfirstline}\strut
\end{minipage} & \begin{minipage}[t]{0.10\columnwidth}\raggedright\strut
value of \texttt{\%input} that is at the start of this conversation
start\strut
\end{minipage}\tabularnewline
\begin{minipage}[t]{0.26\columnwidth}\raggedright\strut
\texttt{\%speaker}\strut
\end{minipage} & \begin{minipage}[t]{0.10\columnwidth}\raggedright\strut
value of \texttt{speaker} from a conversation involving :tsvsource\strut
\end{minipage}\tabularnewline
\begin{minipage}[t]{0.26\columnwidth}\raggedright\strut
\texttt{\%userinput}\strut
\end{minipage} & \begin{minipage}[t]{0.10\columnwidth}\raggedright\strut
Boolean is the current input from the user (vs the chatbot)\strut
\end{minipage}\tabularnewline
\begin{minipage}[t]{0.26\columnwidth}\raggedright\strut
\texttt{\%voice}\strut
\end{minipage} & \begin{minipage}[t]{0.10\columnwidth}\raggedright\strut
active or passive on current input\strut
\end{minipage}\tabularnewline
\begin{minipage}[t]{0.26\columnwidth}\raggedright\strut
\texttt{\%trace\_on}\strut
\end{minipage} & \begin{minipage}[t]{0.10\columnwidth}\raggedright\strut
Fake empty variable used to turn on tracing (see Debugging
commands)\strut
\end{minipage}\tabularnewline
\begin{minipage}[t]{0.26\columnwidth}\raggedright\strut
\texttt{\%trace\_off}\strut
\end{minipage} & \begin{minipage}[t]{0.10\columnwidth}\raggedright\strut
Fake empty variable used to turn off tracing (see Debugging
commands)\strut
\end{minipage}\tabularnewline
\begin{minipage}[t]{0.26\columnwidth}\raggedright\strut
\texttt{\%starttimems}\strut
\end{minipage} & \begin{minipage}[t]{0.10\columnwidth}\raggedright\strut
Start of user request time/date in milliseconds\strut
\end{minipage}\tabularnewline
\begin{minipage}[t]{0.26\columnwidth}\raggedright\strut
\texttt{\%inputsize}\strut
\end{minipage} & \begin{minipage}[t]{0.10\columnwidth}\raggedright\strut
gives how many characters were passed in input\strut
\end{minipage}\tabularnewline
\begin{minipage}[t]{0.26\columnwidth}\raggedright\strut
\texttt{\%inputlimited}\strut
\end{minipage} & \begin{minipage}[t]{0.10\columnwidth}\raggedright\strut
1 if too many characters were given (relative to fullinputlimit)\strut
\end{minipage}\tabularnewline
\begin{minipage}[t]{0.26\columnwidth}\raggedright\strut
\texttt{\%tsvsource}\strut
\end{minipage} & \begin{minipage}[t]{0.10\columnwidth}\raggedright\strut
1 if in progress Null otherwise\strut
\end{minipage}\tabularnewline
\begin{minipage}[t]{0.26\columnwidth}\raggedright\strut
\texttt{\%heapsize}\strut
\end{minipage} & \begin{minipage}[t]{0.10\columnwidth}\raggedright\strut
how many bytes of heap are left\strut
\end{minipage}\tabularnewline
\bottomrule
\end{longtable}

\subsection{Chatbot Output}\label{chatbot-output}

\begin{longtable}[]{@{}ll@{}}
\toprule
\begin{minipage}[b]{0.12\columnwidth}\raggedright\strut
variable\strut
\end{minipage} & \begin{minipage}[b]{0.17\columnwidth}\raggedright\strut
description\strut
\end{minipage}\tabularnewline
\midrule
\endhead
\begin{minipage}[t]{0.12\columnwidth}\raggedright\strut
\texttt{\%inputrejoinder}\strut
\end{minipage} & \begin{minipage}[t]{0.17\columnwidth}\raggedright\strut
rule tag of any pending rejoinder for input or null if none
pending\strut
\end{minipage}\tabularnewline
\begin{minipage}[t]{0.12\columnwidth}\raggedright\strut
\texttt{\%lastoutput}\strut
\end{minipage} & \begin{minipage}[t]{0.17\columnwidth}\raggedright\strut
the text of the last generated response for the current volley - always
null across volleys\strut
\end{minipage}\tabularnewline
\begin{minipage}[t]{0.12\columnwidth}\raggedright\strut
\texttt{\%lastquestion}\strut
\end{minipage} & \begin{minipage}[t]{0.17\columnwidth}\raggedright\strut
Boolean did last output end in a ?\strut
\end{minipage}\tabularnewline
\begin{minipage}[t]{0.12\columnwidth}\raggedright\strut
\texttt{\%outputrejoinder}\strut
\end{minipage} & \begin{minipage}[t]{0.17\columnwidth}\raggedright\strut
rule tag if system set a rejoinder for its current output or 0\strut
\end{minipage}\tabularnewline
\begin{minipage}[t]{0.12\columnwidth}\raggedright\strut
\texttt{\%response}\strut
\end{minipage} & \begin{minipage}[t]{0.17\columnwidth}\raggedright\strut
number of committed responses that have been generated for this sentence
(see Advanced User- Advanced Output: Committed Responses\strut
\end{minipage}\tabularnewline
\bottomrule
\end{longtable}

\subsection{System variables}\label{system-variables-1}

Note for all time variables, they normally use local machine time. If
you have a \$cs\_utcoffset variable with a value, then all time will be
relative to GMT/UTC/Zulu (which means it doesn't pay attention to
daylight savings and you have to do that yourself with the answer).

\begin{longtable}[]{@{}ll@{}}
\toprule
\begin{minipage}[b]{0.12\columnwidth}\raggedright\strut
variable\strut
\end{minipage} & \begin{minipage}[b]{0.10\columnwidth}\raggedright\strut
description\strut
\end{minipage}\tabularnewline
\midrule
\endhead
\begin{minipage}[t]{0.12\columnwidth}\raggedright\strut
\texttt{\%all}\strut
\end{minipage} & \begin{minipage}[t]{0.10\columnwidth}\raggedright\strut
Boolean is the :all flag on? (:all to set)\strut
\end{minipage}\tabularnewline
\begin{minipage}[t]{0.12\columnwidth}\raggedright\strut
\texttt{\%document}\strut
\end{minipage} & \begin{minipage}[t]{0.10\columnwidth}\raggedright\strut
Boolean is :document running\strut
\end{minipage}\tabularnewline
\begin{minipage}[t]{0.12\columnwidth}\raggedright\strut
\texttt{\%fact}\strut
\end{minipage} & \begin{minipage}[t]{0.10\columnwidth}\raggedright\strut
Numeric value most recent fact id\strut
\end{minipage}\tabularnewline
\begin{minipage}[t]{0.12\columnwidth}\raggedright\strut
\texttt{\%freetext}\strut
\end{minipage} & \begin{minipage}[t]{0.10\columnwidth}\raggedright\strut
kb of available text space\strut
\end{minipage}\tabularnewline
\begin{minipage}[t]{0.12\columnwidth}\raggedright\strut
\texttt{\%freedict}\strut
\end{minipage} & \begin{minipage}[t]{0.10\columnwidth}\raggedright\strut
number of unused dictionary words\strut
\end{minipage}\tabularnewline
\begin{minipage}[t]{0.12\columnwidth}\raggedright\strut
\texttt{\%freefact}\strut
\end{minipage} & \begin{minipage}[t]{0.10\columnwidth}\raggedright\strut
number of unused facts\strut
\end{minipage}\tabularnewline
\begin{minipage}[t]{0.12\columnwidth}\raggedright\strut
\texttt{\%maxmatchvariables}\strut
\end{minipage} & \begin{minipage}[t]{0.10\columnwidth}\raggedright\strut
highest number of match variables, currently 20\strut
\end{minipage}\tabularnewline
\begin{minipage}[t]{0.12\columnwidth}\raggedright\strut
\texttt{\%maxfactsets}\strut
\end{minipage} & \begin{minipage}[t]{0.10\columnwidth}\raggedright\strut
highest number of @factsets, currently 20\strut
\end{minipage}\tabularnewline
\begin{minipage}[t]{0.12\columnwidth}\raggedright\strut
\texttt{\%host}\strut
\end{minipage} & \begin{minipage}[t]{0.10\columnwidth}\raggedright\strut
name of the current host machine or ``local''\strut
\end{minipage}\tabularnewline
\begin{minipage}[t]{0.12\columnwidth}\raggedright\strut
\texttt{\%regression}\strut
\end{minipage} & \begin{minipage}[t]{0.10\columnwidth}\raggedright\strut
Boolean is the regression flag on\strut
\end{minipage}\tabularnewline
\begin{minipage}[t]{0.12\columnwidth}\raggedright\strut
\texttt{\%server}\strut
\end{minipage} & \begin{minipage}[t]{0.10\columnwidth}\raggedright\strut
Boolean is the system running in server mode\strut
\end{minipage}\tabularnewline
\begin{minipage}[t]{0.12\columnwidth}\raggedright\strut
\texttt{\%rule}\strut
\end{minipage} & \begin{minipage}[t]{0.10\columnwidth}\raggedright\strut
get a tag to the current executing rule. Can be used in place of a
label\strut
\end{minipage}\tabularnewline
\begin{minipage}[t]{0.12\columnwidth}\raggedright\strut
\texttt{\%topic}\strut
\end{minipage} & \begin{minipage}[t]{0.10\columnwidth}\raggedright\strut
name of the current ``real'' topic . if control is currently in a topic
or called from a topic which is not system or nostay, then that is the
topic. Otherwise the most recent pending topic is found\strut
\end{minipage}\tabularnewline
\begin{minipage}[t]{0.12\columnwidth}\raggedright\strut
\texttt{\%actualtopic}\strut
\end{minipage} & \begin{minipage}[t]{0.10\columnwidth}\raggedright\strut
literally the current topic being processed (system or not)\strut
\end{minipage}\tabularnewline
\begin{minipage}[t]{0.12\columnwidth}\raggedright\strut
\texttt{\%trace}\strut
\end{minipage} & \begin{minipage}[t]{0.10\columnwidth}\raggedright\strut
Numeric value of the trace flag (:trace to set)\strut
\end{minipage}\tabularnewline
\begin{minipage}[t]{0.12\columnwidth}\raggedright\strut
\texttt{\%httpresponse}\strut
\end{minipage} & \begin{minipage}[t]{0.10\columnwidth}\raggedright\strut
return code of most recent \^{}jsonopen call\strut
\end{minipage}\tabularnewline
\begin{minipage}[t]{0.12\columnwidth}\raggedright\strut
\texttt{\%pid}\strut
\end{minipage} & \begin{minipage}[t]{0.10\columnwidth}\raggedright\strut
Linux process id or 0 for other systems\strut
\end{minipage}\tabularnewline
\begin{minipage}[t]{0.12\columnwidth}\raggedright\strut
\texttt{\%restart}\strut
\end{minipage} & \begin{minipage}[t]{0.10\columnwidth}\raggedright\strut
You can set and retrieve a value here across a system restart.\strut
\end{minipage}\tabularnewline
\begin{minipage}[t]{0.12\columnwidth}\raggedright\strut
\texttt{\%timeout}\strut
\end{minipage} & \begin{minipage}[t]{0.10\columnwidth}\raggedright\strut
Boolean tells if a timeout has happened, based on the timelimit command
line parameter\strut
\end{minipage}\tabularnewline
\begin{minipage}[t]{0.12\columnwidth}\raggedright\strut
\texttt{\%lastcurltime}\strut
\end{minipage} & \begin{minipage}[t]{0.10\columnwidth}\raggedright\strut
Time Analysis: Name Look up: Host/proxy connect: App(SSL) connect:
Pretransfer: Total Transfer:\strut
\end{minipage}\tabularnewline
\begin{minipage}[t]{0.12\columnwidth}\raggedright\strut
\texttt{\%crosstalk}\strut
\end{minipage} & \begin{minipage}[t]{0.10\columnwidth}\raggedright\strut
4k buffer in server visible between users to pass data back and
forth\strut
\end{minipage}\tabularnewline
\begin{minipage}[t]{0.12\columnwidth}\raggedright\strut
\texttt{\%crosstalk1}\strut
\end{minipage} & \begin{minipage}[t]{0.10\columnwidth}\raggedright\strut
4k buffer in server visible between users to pass data back and
forth\strut
\end{minipage}\tabularnewline
\begin{minipage}[t]{0.12\columnwidth}\raggedright\strut
\texttt{\%crosstalk2}\strut
\end{minipage} & \begin{minipage}[t]{0.10\columnwidth}\raggedright\strut
4k buffer in server visible between users to pass data back and
forth\strut
\end{minipage}\tabularnewline
\begin{minipage}[t]{0.12\columnwidth}\raggedright\strut
\texttt{\%crosstalk3}\strut
\end{minipage} & \begin{minipage}[t]{0.10\columnwidth}\raggedright\strut
4k buffer in server visible between users to pass data back and
forth\strut
\end{minipage}\tabularnewline
\begin{minipage}[t]{0.12\columnwidth}\raggedright\strut
\texttt{\%logging}\strut
\end{minipage} & \begin{minipage}[t]{0.10\columnwidth}\raggedright\strut
bit status of serverLog, userLog, and host name - 0=off 1=file 2= stdout
4=stderr 8=prelog)\strut
\end{minipage}\tabularnewline
\begin{minipage}[t]{0.12\columnwidth}\raggedright\strut
\texttt{\%forkcount}\strut
\end{minipage} & \begin{minipage}[t]{0.10\columnwidth}\raggedright\strut
number of forks requested in linux evserver environment\strut
\end{minipage}\tabularnewline
\begin{minipage}[t]{0.12\columnwidth}\raggedright\strut
\texttt{\%dbparams}\strut
\end{minipage} & \begin{minipage}[t]{0.10\columnwidth}\raggedright\strut
copy of the server params given to db used as fileserver (pg or mysql or
mssql or mongo)\strut
\end{minipage}\tabularnewline
\begin{minipage}[t]{0.12\columnwidth}\raggedright\strut
\texttt{\%botid}\strut
\end{minipage} & \begin{minipage}[t]{0.10\columnwidth}\raggedright\strut
bot id number in use\strut
\end{minipage}\tabularnewline
\begin{minipage}[t]{0.12\columnwidth}\raggedright\strut
\texttt{\%curlversion}\strut
\end{minipage} & \begin{minipage}[t]{0.10\columnwidth}\raggedright\strut
curl version information\strut
\end{minipage}\tabularnewline
\begin{minipage}[t]{0.12\columnwidth}\raggedright\strut
\texttt{\%dbversion}\strut
\end{minipage} & \begin{minipage}[t]{0.10\columnwidth}\raggedright\strut
db version information\strut
\end{minipage}\tabularnewline
\begin{minipage}[t]{0.12\columnwidth}\raggedright\strut
\texttt{\%testpattern}\strut
\end{minipage} & \begin{minipage}[t]{0.10\columnwidth}\raggedright\strut
The index number in the array of patterns of current pattern being
matched in \^{}testpattern\strut
\end{minipage}\tabularnewline
\bottomrule
\end{longtable}

\subsection{\^{}testpattern control
variables}\label{testpattern-control-variables}

\texttt{\%testpattern-nosave} \textbar{} blocks saving NL from
\^{}testpattern if nlsave=1 was set in command line params\\
\texttt{\%testpattern-prescan} \textbar{} execute this pattern on all
sentences before doing other patterns one-by-one\\
\texttt{\%trace\_on} \textbar{} starting here, do :trace pattern in
\^{}testpattern\\
\texttt{\%trace\_on\ all} \textbar{} starting here, do :trace all in
\^{}testpattern\\
\texttt{\%trace\_off} \textbar{} turn off tracing (also turns off at end
of cs call)

\subsection{Build data}\label{build-data}

\begin{longtable}[]{@{}ll@{}}
\toprule
variable & description\tabularnewline
\midrule
\endhead
\texttt{\%dict} & date/time the dictionary was built\tabularnewline
\texttt{\%engine} & date/time the engine was compiled\tabularnewline
\texttt{\%os} & os invovled (linux windows mac ios)\tabularnewline
\texttt{\%script} & date/time build1 was compiled\tabularnewline
\texttt{\%version} & engine version number\tabularnewline
\bottomrule
\end{longtable}

You actually can assign to any of them. This will override them and make
them return what you tell them to and is a particularly BAD thing to do
if this is running on a server since it affects all users (unless you
reset the variable at the end of the volley. Assigning a period to a
variable resets it).

Typically one does this as a temporary assignment in a \texttt{\#!}
comment line to set up conditions for testing using \texttt{:verify}.

Making them return a new value is NOT the same thing as making the
engine have a different value. Unless the variable is marked as
settable, setting a value affects only the value returned by a future
call to the system variable. It does not change engine values the
variable is meant to reflect.

\section{Control Over Input}\label{control-over-input}

The system can do a number of standard processing on user input,
including spell correction, proper-name merging, expanding contractions
etc. This is managed by setting the user variable \texttt{\$cs\_token}.

The default \$cs\_token that comes with Harry is:

\begin{verbatim}
$cs_token = #DO_INTERJECTION_SPLITTING | 
            #DO_SUBSTITUTE_SYSTEM |
            #DO_NUMBER_MERGE | 
            #DO_PROPERNAME_MERGE | 
            #DO_SPELLCHECK |
            #DO_PARSE
\end{verbatim}

The \texttt{\#}signals a named constant from the
\texttt{dictionarySystem.h} file. One can set the following:

These enable various LIVEDATA files to perform substitutions on input:

\begin{longtable}[]{@{}ll@{}}
\toprule
\begin{minipage}[b]{0.31\columnwidth}\raggedright\strut
flag\strut
\end{minipage} & \begin{minipage}[b]{0.63\columnwidth}\raggedright\strut
description\strut
\end{minipage}\tabularnewline
\midrule
\endhead
\begin{minipage}[t]{0.31\columnwidth}\raggedright\strut
\texttt{\#DO\_ESSENTIALS}\strut
\end{minipage} & \begin{minipage}[t]{0.63\columnwidth}\raggedright\strut
perform \texttt{LIVEDATA/systemessentials} which mostly strips off
trailing punctuation and sets corresponding flags instead\strut
\end{minipage}\tabularnewline
\begin{minipage}[t]{0.31\columnwidth}\raggedright\strut
\texttt{\#DO\_SUBSTITUTES}\strut
\end{minipage} & \begin{minipage}[t]{0.63\columnwidth}\raggedright\strut
perform \texttt{LIVEDATA/substitutes}\strut
\end{minipage}\tabularnewline
\begin{minipage}[t]{0.31\columnwidth}\raggedright\strut
\texttt{\#DO\_CONTRACTIONS}\strut
\end{minipage} & \begin{minipage}[t]{0.63\columnwidth}\raggedright\strut
perform \texttt{LIVEDATA/contractions}, expanding contractions\strut
\end{minipage}\tabularnewline
\begin{minipage}[t]{0.31\columnwidth}\raggedright\strut
\texttt{\#DO\_INTERJECTIONS}\strut
\end{minipage} & \begin{minipage}[t]{0.63\columnwidth}\raggedright\strut
perform \texttt{LIVEDATA/interjections}, changing phrases to
interjections\strut
\end{minipage}\tabularnewline
\begin{minipage}[t]{0.31\columnwidth}\raggedright\strut
\texttt{\#DO\_BRITISH}\strut
\end{minipage} & \begin{minipage}[t]{0.63\columnwidth}\raggedright\strut
perform \texttt{LIVEDATA/british}, respelling brit words to
American\strut
\end{minipage}\tabularnewline
\begin{minipage}[t]{0.31\columnwidth}\raggedright\strut
\texttt{\#DO\_SPELLING}\strut
\end{minipage} & \begin{minipage}[t]{0.63\columnwidth}\raggedright\strut
performs the \texttt{LIVEDATA/spelling} file (manual spell
correction)\strut
\end{minipage}\tabularnewline
\begin{minipage}[t]{0.31\columnwidth}\raggedright\strut
\texttt{\#DO\_TEXTING}\strut
\end{minipage} & \begin{minipage}[t]{0.63\columnwidth}\raggedright\strut
performs the \texttt{LIVEDATA/texting} file (expand texting
notation)\strut
\end{minipage}\tabularnewline
\begin{minipage}[t]{0.31\columnwidth}\raggedright\strut
\texttt{\#DO\_SUBSTITUTE\_SYSTEM}\strut
\end{minipage} & \begin{minipage}[t]{0.63\columnwidth}\raggedright\strut
do all LIVEDATA file expansions\strut
\end{minipage}\tabularnewline
\bottomrule
\end{longtable}

The contents of the files above are pairs of tokens per line. Left is
the word to replace and right is the replacement. When multiple words
are involved, the left side uses underscores to represent this and the
right side uses \texttt{+}. If the right side is missing, it means just
delete. \textbar{} \texttt{\#DO\_INTERJECTION\_SPLITTING} \textbar{}
break off leading interjections into own sentence \textbar{}
\texttt{\#\$DO\_NUMBER\_MERGE} \textbar{} merge multiple word numbers
into one (\emph{four and twenty})\\
\textbar{} \texttt{\#\$DO\_PROPERNAME\_MERGE} \textbar{} merge multiple
proper name into one (\emph{George Harrison}) \textbar{}
\texttt{\#DO\_DATE\_MERGE} \textbar{} merge month day and/or year
sequences (\emph{January 2, 1993}) \textbar{}
\texttt{\#JSON\_DIRECT\_FROM\_OOB} \textbar{} asking the tokenizer to
directly process OOB data. See \texttt{\^{}jsonparse} in JSON manual.
\textbar{} \texttt{\#NO\_FIX\_UTF} \textbar{} do not adjust inputs with
html or utf8 encodings to simple ascii.\\
\textbar{} \texttt{\#TOKENIZE\_BY\_CHARACTER} \textbar{} Every
non-whitespace character becomes its own token and canonical form. (good
for Japanese)

If any of the above items affect the input (except
TOKENIZE\_BY\_CHARACTER), they will be echoed as values into
\texttt{\%tokenFlags} so you can detect they happened. The next changes
do not echo into \%tokenFlags and relate to grammar of input:

\begin{longtable}[]{@{}ll@{}}
\toprule
\begin{minipage}[b]{0.28\columnwidth}\raggedright\strut
flag\strut
\end{minipage} & \begin{minipage}[b]{0.67\columnwidth}\raggedright\strut
description\strut
\end{minipage}\tabularnewline
\midrule
\endhead
\begin{minipage}[t]{0.28\columnwidth}\raggedright\strut
\texttt{DO\_POSTAG}\strut
\end{minipage} & \begin{minipage}[t]{0.67\columnwidth}\raggedright\strut
allow pos-tagging (labels like \textasciitilde{}noun
\textasciitilde{}verb become marked)\strut
\end{minipage}\tabularnewline
\begin{minipage}[t]{0.28\columnwidth}\raggedright\strut
\texttt{DO\_PARSE}\strut
\end{minipage} & \begin{minipage}[t]{0.67\columnwidth}\raggedright\strut
allow parser (labels for word roles like
\textasciitilde{}main\_subject)\strut
\end{minipage}\tabularnewline
\begin{minipage}[t]{0.28\columnwidth}\raggedright\strut
\texttt{DO\_CONDITIONAL\_POSTAG}\strut
\end{minipage} & \begin{minipage}[t]{0.67\columnwidth}\raggedright\strut
perform pos-tagging only if all words are known. Avoids wasting time on
foreign sentences in particular\strut
\end{minipage}\tabularnewline
\begin{minipage}[t]{0.28\columnwidth}\raggedright\strut
\texttt{NO\_CONDITIONAL\_IDIOM}\strut
\end{minipage} & \begin{minipage}[t]{0.67\columnwidth}\raggedright\strut
will not perform substitutions in the dictionary which are considered
conditional idioms\strut
\end{minipage}\tabularnewline
\begin{minipage}[t]{0.28\columnwidth}\raggedright\strut
\texttt{NO\_ERASE}\strut
\end{minipage} & \begin{minipage}[t]{0.67\columnwidth}\raggedright\strut
where a substitution would delete a word entirely as junk, don't\strut
\end{minipage}\tabularnewline
\begin{minipage}[t]{0.28\columnwidth}\raggedright\strut
\texttt{DO\_SPLIT\_UNDERSCORES}\strut
\end{minipage} & \begin{minipage}[t]{0.67\columnwidth}\raggedright\strut
happens after all other input tokenization and adjustments except number
merge, and separates words that have been conjoined either because the
dictionary has them (\emph{credit\_card}) or because they were merged by
proper name merging, or by substitution. The result is only words
without underscores (excluding number words like
\emph{five\_thousand\_and\_four}\strut
\end{minipage}\tabularnewline
\begin{minipage}[t]{0.28\columnwidth}\raggedright\strut
\texttt{MARK\_LOWER}\strut
\end{minipage} & \begin{minipage}[t]{0.67\columnwidth}\raggedright\strut
if a word is considered a proper name in CS and is marked as an upper
case word, this will force it to perform any markings for its lower case
form as well. Sometimes users type stuff in upper case that really
should be lower\strut
\end{minipage}\tabularnewline
\bottomrule
\end{longtable}

Normally the system tries to outguess the user, who cannot be trusted to
use correct punctuation or casing or spelling. These block that:

\begin{longtable}[]{@{}ll@{}}
\toprule
\begin{minipage}[b]{0.07\columnwidth}\raggedright\strut
flag\strut
\end{minipage} & \begin{minipage}[b]{0.10\columnwidth}\raggedright\strut
description\strut
\end{minipage}\tabularnewline
\midrule
\endhead
\begin{minipage}[t]{0.07\columnwidth}\raggedright\strut
\texttt{STRICT\_CASING}\strut
\end{minipage} & \begin{minipage}[t]{0.10\columnwidth}\raggedright\strut
except for 1st word of a sentence, assume user uses correct casing on
words\strut
\end{minipage}\tabularnewline
\begin{minipage}[t]{0.07\columnwidth}\raggedright\strut
\texttt{NO\_INFER\_QUESTION}\strut
\end{minipage} & \begin{minipage}[t]{0.10\columnwidth}\raggedright\strut
the system will not try to set the QUESTIONMARK flag if the user didn't
input a ? and the structure of the input looks like a question\strut
\end{minipage}\tabularnewline
\begin{minipage}[t]{0.07\columnwidth}\raggedright\strut
\texttt{DO\_SPELLCHECK}\strut
\end{minipage} & \begin{minipage}[t]{0.10\columnwidth}\raggedright\strut
perform internal spell checking\strut
\end{minipage}\tabularnewline
\begin{minipage}[t]{0.07\columnwidth}\raggedright\strut
\texttt{ONLY\_LOWERCASE}\strut
\end{minipage} & \begin{minipage}[t]{0.10\columnwidth}\raggedright\strut
force all input (except ``I'') to be lower case, refuse to recognize
uppercase forms of anything\strut
\end{minipage}\tabularnewline
\begin{minipage}[t]{0.07\columnwidth}\raggedright\strut
\texttt{NO\_IMPERATIVE}\strut
\end{minipage} & \begin{minipage}[t]{0.10\columnwidth}\raggedright\strut
\strut
\end{minipage}\tabularnewline
\begin{minipage}[t]{0.07\columnwidth}\raggedright\strut
\texttt{NO\_WITHIN}\strut
\end{minipage} & \begin{minipage}[t]{0.10\columnwidth}\raggedright\strut
don't match fragments within a composite word\strut
\end{minipage}\tabularnewline
\begin{minipage}[t]{0.07\columnwidth}\raggedright\strut
\texttt{NO\_SENTENCE\_END}\strut
\end{minipage} & \begin{minipage}[t]{0.10\columnwidth}\raggedright\strut
do not break input into sentences\strut
\end{minipage}\tabularnewline
\bottomrule
\end{longtable}

Normally the tokenizer breaks apart some kinds of sentences into two.
These prevent that:

\begin{longtable}[]{@{}ll@{}}
\toprule
\begin{minipage}[b]{0.07\columnwidth}\raggedright\strut
flag\strut
\end{minipage} & \begin{minipage}[b]{0.10\columnwidth}\raggedright\strut
description\strut
\end{minipage}\tabularnewline
\midrule
\endhead
\begin{minipage}[t]{0.07\columnwidth}\raggedright\strut
\texttt{NO\_COLON\_END}\strut
\end{minipage} & \begin{minipage}[t]{0.10\columnwidth}\raggedright\strut
don't break apart a sentence after a colon\strut
\end{minipage}\tabularnewline
\begin{minipage}[t]{0.07\columnwidth}\raggedright\strut
\texttt{NO\_SEMICOLON\_END}\strut
\end{minipage} & \begin{minipage}[t]{0.10\columnwidth}\raggedright\strut
don't break apart a sentence after a semi-colon\strut
\end{minipage}\tabularnewline
\begin{minipage}[t]{0.07\columnwidth}\raggedright\strut
\texttt{UNTOUCHED\_INPUT}\strut
\end{minipage} & \begin{minipage}[t]{0.10\columnwidth}\raggedright\strut
if set to this alone, will tokenize only on spaces, leaving everything
but spacing untouched\strut
\end{minipage}\tabularnewline
\begin{minipage}[t]{0.07\columnwidth}\raggedright\strut
\texttt{LEAVE\_QUOTE}\strut
\end{minipage} & \begin{minipage}[t]{0.10\columnwidth}\raggedright\strut
if input is found within " " it will become a single token exactly as it
is seen. W/o Leave\_Quote, it is converted into a word without quotes
and using underscores instead of spaces. So ``My Fair Lady'' becomes
My\_Fair\_Lady, which would match a movie title if you had one, unlike
\emph{My Fair Lady} becoming the resulting token and unrecognized\strut
\end{minipage}\tabularnewline
\begin{minipage}[t]{0.07\columnwidth}\raggedright\strut
\texttt{SPLIT\_QUOTE}\strut
\end{minipage} & \begin{minipage}[t]{0.10\columnwidth}\raggedright\strut
if input is found within " " the quotes will be removed.\strut
\end{minipage}\tabularnewline
\bottomrule
\end{longtable}

Note

you can change \texttt{\$cs\_token} on the fly and force input to be
reanalyzed via \texttt{\^{}retry(SENTENCE)}. I do this when I detect the
user is trying to give his name, and many foreign names might be
spell-corrected into something wrong and the user is unlikely to
misspell his own name.

Just remember to reset \texttt{\$cs\_token} back to normal after you are
done. Here is one such way, assuming \texttt{\$stdtoken} is set to your
normal tokenflags in your bot definition outputmacro:

\begin{verbatim}
#! my name is Rogr
s: (name is _*)

    if ($cs_token == $stdtoken)
        {
        $cs_token = #DO_INTERJECTION_SPLITTING |
                    #DO_SUBSTITUTE_SYSTEM | #DO_NUMBER_MERGE |
                    #DO_PARSE
        retry(SENTENCE)
        }
    _0 is the name.
    $cs_token = $stdtoken
\end{verbatim}

If you type \emph{my name is Rogr} into a topic with this, the original
input is spell-corrected to \emph{my name is Roger}, but this will
change the \texttt{\$cs\_token} over to one without spell correction and
redo the sentence, which will now come back with \emph{my name is Rogr}
and be echoed correctly, and \texttt{\$cs\_token\ reset}.

That's assuming nothing else would run differently and trap the response
elsewhere. If you were worried about that, it would be possible for the
script to save where it is using \texttt{\^{}getrule(tag)} and modify
your control script to return immediate control to here after input
processing if you had changed \texttt{\$cs\_token}.

\subsection{\%tokenflags}\label{tokenflags}

These are the values that \%tokenflags may have after analysis of a
sentence\ldots{} \#define PRESENT 0x0000000000002000ULL\\
\#define PAST 0x0000000000004000ULL // basic tense- both present perfect
and past perfect map to it \#define FUTURE 0x0000000000008000ULL\\
\#define PRESENT\_PERFECT 0x0000000000010000ULL // distinguish PAST
PERFECT from PAST PRESENT\_PERFECT \#define CONTINUOUS
0x0000000000020000ULL\\
\#define PERFECT 0x0000000000040000ULL\\
\#define PASSIVE 0x0000000000080000ULL

\section{define IMPLIED\_SUBJECT}\label{define-implied_subject}

\section{define QUESTIONMARK}\label{define-questionmark}

\section{define EXCLAMATIONMARK}\label{define-exclamationmark}

\section{define PERIODMARK}\label{define-periodmark}

\section{define USERINPUT}\label{define-userinput}

\section{define COMMANDMARK}\label{define-commandmark}

\section{define IMPLIED\_YOU}\label{define-implied_you}

\section{FOREIGN\_TOKENS}\label{foreign_tokens}

\section{FAULTY\_PARSE}\label{faulty_parse}

\section{QUOTATION}\label{quotation}

\section{NOT\_SENTENCE}\label{not_sentence}

One or more of these will be set if input was changed do to use of these
files

\begin{verbatim}
#DO_ESSENTIALS           
#DO_SUBSTITUTES         
#DO_CONTRACTIONS            
#DO_INTERJECTIONS       
#DO_BRITISH              
#DO_SPELLING                 
#DO_TEXTING              
#DO_NOISE                   
#DO_PRIVATE                 
#DO_NUMBER_MERGE                
#DO_PROPERNAME_MERGE            
#DO_SPELLCHECK                  
#DO_INTERJECTION_SPLITTING    
\end{verbatim}

\subsection{Private Substitutions}\label{private-substitutions}

While in general, substitutions are defined in the LIVEDATA folder, you
can define private substititions for your specific bot using the
scripting language. You can say

\begin{verbatim}
replace: xxx yyyyy
\end{verbatim}

which defines a substitution just like a livedata substitution file. It
actually creates a substitution file called \texttt{private0.txt} or
\texttt{private1.txt} in your TOPIC folder.

Even then, those substitutions will not be enacted unless you explicitly
add to the \texttt{\$cs\_token} value \texttt{\#DO\_PRIVATE}, eg

\begin{verbatim}
$cs_token = #DO_INTERJECTION_SPLITTING | 
            #DO_SUBSTITUTE_SYSTEM |
            #DO_NUMBER_MERGE | 
            #DO_PROPERNAME_MERGE |
            #DO_SPELLCHECK | 
            #DO_PARSE | 
            #DO_PRIVATE
\end{verbatim}

The left side of the substitution pair is case insensitive (matches
either case on input) and can be placed in double-quotes (which converts
spaces to underscores internally).

The right side of the substitution pair is case sensitive and can be
placed in double-quotes (which converts spaces to plus signs
internally).

Note: if you privately define a substitution that leads to a known
interjection, it will be treated as an interjection, marked as
DO\_INTERJECTIONS rather than DO\_PRIVATE. Interjections do not perform
an actual substitution, does not replace the words on the left with the
interjection concept name on the right. Instead interjections merely
mark the phrase as being a member of that concept, leaving the actual
words unchanged.

Similarly while canonical values of words can be defined in
\texttt{LIVEDATA/SYSTEM/canonical.txt}, you can define private canonical
values for your bots by using the scripting language. You can say:

\begin{verbatim}
canon: oh 0 
canon: faster fast
\end{verbatim}

which defines new canonical values for things and creates a file
\texttt{canon0.txt} or \texttt{canon1.txt} in your TOPIC folder.

You can optionally add MORE\_FORM or MOST\_FORM as a 3rd argument, to
set those flags for adjectives and adverbs.

If you want to set a canonical pair from a table during compilation, you
can use a function to do the same thing (but only 1 pair at a time).

\begin{verbatim}
^canon(word canonicalform)
\end{verbatim}

\subsection{Numeric Substitutions}\label{numeric-substitutions}

A special kind of private substitution (equally applicable in regular
substitution files) is the numeric substitution.

\begin{verbatim}
replace: ?_km kilometers
\end{verbatim}

The ?\_ matches a digit number followed immediately by km, like
\texttt{1.2km} and will separate the number and replace the units with
the given replacement. The input can be singular or have an `s' like
\texttt{10.5dollars}. And it can be with or without abbreviation
periods, like \texttt{10kps} or \texttt{10k.p.s}

\subsection{Apostrophe Substitutions
replace}\label{apostrophe-substitutions-replace}

\begin{verbatim}
replace: 'xxx  yyy
\end{verbatim}

allows you to split during tokenization any word followed by 'xxx into
two words, original sans 'xxx and yyy. eg

\begin{verbatim}
replace: 've have
\end{verbatim}

gives ``companies've =\textgreater{}''companies have``.

\subsection{Replacing to a word with + in
it}\label{replacing-to-a-word-with-in-it}

Normally \texttt{replace:\ \ x\ \ y+z} will generate 2 words, y and z.
If you need a plus in your word, you can escape your 2nd word:

\begin{verbatim}
    replace: "black and decker" \BLACK+DECKER
\end{verbatim}

\section{Interchange Variables}\label{interchange-variables}

The following variables can be defined in a script and the engine will
react to their contents.

\begin{longtable}[]{@{}ll@{}}
\toprule
\begin{minipage}[b]{0.26\columnwidth}\raggedright\strut
interchange variable\strut
\end{minipage} & \begin{minipage}[b]{0.10\columnwidth}\raggedright\strut
description\strut
\end{minipage}\tabularnewline
\midrule
\endhead
\begin{minipage}[t]{0.26\columnwidth}\raggedright\strut
\texttt{\$cs\_token}\strut
\end{minipage} & \begin{minipage}[t]{0.10\columnwidth}\raggedright\strut
described extensively above\strut
\end{minipage}\tabularnewline
\begin{minipage}[t]{0.26\columnwidth}\raggedright\strut
\texttt{\$cs\_response}\strut
\end{minipage} & \begin{minipage}[t]{0.10\columnwidth}\raggedright\strut
controls automatic handling of outputs to user. By default it consists
of
\texttt{\$cs\_response\ =\ \#Response\_upperstart\ \textbar{}\ \#response\_removespacebeforecomma\ \textbar{}\ \#response\_alterunderscores\ \textbar{}\ \#response\_removetilde}
If you want none of theses, use \$cs\_response = 0 (all flags turned
off). See \^{}print for explanation of flags.
\texttt{\#response\_noconvertspecial} - leave escaped n r and t alone in
output and \^{}log, \texttt{\#response\_upperstart} - makes the first
letter of an output sentence capitalized,
\texttt{\#Response\_removespacebeforecomma} - does the obvious,
\texttt{\#Response\_alterunderscores} - converts single underscores to
spaces and double underscores to singles (eg for a web url)\strut
\end{minipage}\tabularnewline
\begin{minipage}[t]{0.26\columnwidth}\raggedright\strut
\texttt{\$cs\_crashmsg}\strut
\end{minipage} & \begin{minipage}[t]{0.10\columnwidth}\raggedright\strut
in server mode, what to say if the server crashes and we return a
message to the user. By default the message is \emph{Hey, sorry. I
forgot what I was thinking about.}\strut
\end{minipage}\tabularnewline
\begin{minipage}[t]{0.26\columnwidth}\raggedright\strut
\texttt{\$cs\_abstract}\strut
\end{minipage} & \begin{minipage}[t]{0.10\columnwidth}\raggedright\strut
used with :abstract\strut
\end{minipage}\tabularnewline
\begin{minipage}[t]{0.26\columnwidth}\raggedright\strut
\texttt{\$cs\_trace}\strut
\end{minipage} & \begin{minipage}[t]{0.10\columnwidth}\raggedright\strut
if this variable is defined, then whenever the user's volley is
finished, the value of this variable is set to that of :trace and :trace
is cleared to 0, but when the user is read back in, the :trace is set to
this value. For a server, this means you can perform tracing on a user
w/o making all user transactions dump trace data\strut
\end{minipage}\tabularnewline
\begin{minipage}[t]{0.26\columnwidth}\raggedright\strut
\texttt{\$cs\_control\_pre}\strut
\end{minipage} & \begin{minipage}[t]{0.10\columnwidth}\raggedright\strut
name of topic (flag it SYSTEM) to run in gambit mode on pre-pass, set by
author. Runs before any sentences of the input volley are analyzed. Good
for setting up initial values\strut
\end{minipage}\tabularnewline
\begin{minipage}[t]{0.26\columnwidth}\raggedright\strut
\texttt{\$cs\_usermessagelimit}\strut
\end{minipage} & \begin{minipage}[t]{0.10\columnwidth}\raggedright\strut
max number of message pairs (user input \& bot output) saved in topic
file\strut
\end{minipage}\tabularnewline
\begin{minipage}[t]{0.26\columnwidth}\raggedright\strut
\texttt{\$cs\_externaltag}\strut
\end{minipage} & \begin{minipage}[t]{0.10\columnwidth}\raggedright\strut
name of a topic to use to replace existing internal English pos-parser.
See bottom of ChatScript PosParser manual for details\strut
\end{minipage}\tabularnewline
\begin{minipage}[t]{0.26\columnwidth}\raggedright\strut
\texttt{\$cs\_prepass}\strut
\end{minipage} & \begin{minipage}[t]{0.10\columnwidth}\raggedright\strut
name of a topic (mark it SYSTEM) to run in responder mode on main
volleys, which runs before \$cs\_control\_main and after all of the
above and pos-parsing is done. Used to amend preparation data coming
from the engine. You can use it to add your own spin on input processing
before going to your main control. I use it to, for example, label
commands as questions, standardize sentence construction (like \emph{if
you see me what will you think} to \emph{assume you see me. What will
you think?})\strut
\end{minipage}\tabularnewline
\begin{minipage}[t]{0.26\columnwidth}\raggedright\strut
\texttt{\$cs\_control\_main}\strut
\end{minipage} & \begin{minipage}[t]{0.10\columnwidth}\raggedright\strut
name of topic (flag it SYSTEM) to run in responder mode on main volleys,
set by author\strut
\end{minipage}\tabularnewline
\begin{minipage}[t]{0.26\columnwidth}\raggedright\strut
\texttt{\$cs\_control\_post}\strut
\end{minipage} & \begin{minipage}[t]{0.10\columnwidth}\raggedright\strut
name of topic (flag it SYSTEM) to run in gambit mode on post-pass, set
by author\strut
\end{minipage}\tabularnewline
\begin{minipage}[t]{0.26\columnwidth}\raggedright\strut
\texttt{\$botprompt}\strut
\end{minipage} & \begin{minipage}[t]{0.10\columnwidth}\raggedright\strut
message for console window to label bot output\strut
\end{minipage}\tabularnewline
\begin{minipage}[t]{0.26\columnwidth}\raggedright\strut
\texttt{\$userprompt}\strut
\end{minipage} & \begin{minipage}[t]{0.10\columnwidth}\raggedright\strut
message for console window to label user input line\strut
\end{minipage}\tabularnewline
\begin{minipage}[t]{0.26\columnwidth}\raggedright\strut
\texttt{\$cs\_crashmsg}\strut
\end{minipage} & \begin{minipage}[t]{0.10\columnwidth}\raggedright\strut
message to use if a crash occurs. see also \$cs\_crash\strut
\end{minipage}\tabularnewline
\begin{minipage}[t]{0.26\columnwidth}\raggedright\strut
\texttt{\$cs\_crash}\strut
\end{minipage} & \begin{minipage}[t]{0.10\columnwidth}\raggedright\strut
topic to execute in gambit mode if a crash occurs. see also
\$cs\_crashmsg\strut
\end{minipage}\tabularnewline
\begin{minipage}[t]{0.26\columnwidth}\raggedright\strut
\texttt{\$cs\_language}\strut
\end{minipage} & \begin{minipage}[t]{0.10\columnwidth}\raggedright\strut
if spanish, will adjust spell checking for spanish colloquial\strut
\end{minipage}\tabularnewline
\begin{minipage}[t]{0.26\columnwidth}\raggedright\strut
\texttt{\$cs\_token}\strut
\end{minipage} & \begin{minipage}[t]{0.10\columnwidth}\raggedright\strut
bits controlling how the tokenizer works. By default when null, you get
all bits assumed on. The possible values are in src/dictionarySystem.h
(hunt for \$token) and you put a \# in front of them to generate that
named numeric constant\strut
\end{minipage}\tabularnewline
\begin{minipage}[t]{0.26\columnwidth}\raggedright\strut
\texttt{\$cs\_abstract}\strut
\end{minipage} & \begin{minipage}[t]{0.10\columnwidth}\raggedright\strut
topic used by :abstract to display facts if you want them
displayed\strut
\end{minipage}\tabularnewline
\begin{minipage}[t]{0.26\columnwidth}\raggedright\strut
\texttt{\$cs\_prepass}\strut
\end{minipage} & \begin{minipage}[t]{0.10\columnwidth}\raggedright\strut
topic used between parsing and running user control script. Useful to
supplement parsing, setting the question value, and revising input
idioms\strut
\end{minipage}\tabularnewline
\begin{minipage}[t]{0.26\columnwidth}\raggedright\strut
\texttt{\$cs\_wildcardseparator}\strut
\end{minipage} & \begin{minipage}[t]{0.10\columnwidth}\raggedright\strut
when a match variable covers multiple words, what should separate them-
by default it's a space, but underscore is handy too. Initial system
character is space, creating fidelity with what was typed. Useful if \_
can be recognized in input (web addresses). Changing to \_ is consistent
with multi-word representation and keyword recognition for concepts. CS
automatically converts \_ to space on output, so internal use of \_ is
normal\strut
\end{minipage}\tabularnewline
\begin{minipage}[t]{0.26\columnwidth}\raggedright\strut
\texttt{\$cs\_userfactlimit}\strut
\end{minipage} & \begin{minipage}[t]{0.10\columnwidth}\raggedright\strut
how many of the most recent permanent facts created by the script in
response to user inputs are kept for each user. Std default is 100. *
means all.\strut
\end{minipage}\tabularnewline
\begin{minipage}[t]{0.26\columnwidth}\raggedright\strut
\texttt{\$cs\_outputchoice}\strut
\end{minipage} & \begin{minipage}[t]{0.10\columnwidth}\raggedright\strut
for regression: forces specific one of a {[}{]} {[}{]} output choice
block - base 0\strut
\end{minipage}\tabularnewline
\begin{minipage}[t]{0.26\columnwidth}\raggedright\strut
\texttt{\$cs\_response}\strut
\end{minipage} & \begin{minipage}[t]{0.10\columnwidth}\raggedright\strut
controls some characteristics of how responses are formatted\strut
\end{minipage}\tabularnewline
\begin{minipage}[t]{0.26\columnwidth}\raggedright\strut
\texttt{\$cs\_randIndex}\strut
\end{minipage} & \begin{minipage}[t]{0.10\columnwidth}\raggedright\strut
the random seed for this volley\strut
\end{minipage}\tabularnewline
\begin{minipage}[t]{0.26\columnwidth}\raggedright\strut
\texttt{\$cs\_utcoffset}\strut
\end{minipage} & \begin{minipage}[t]{0.10\columnwidth}\raggedright\strut
if defined, then \%time returns current utc time + timezone offset. The
offset is usually a simple number, meaning hours, and can have + or - in
front of it. It can also be a normal time reference like 02:30 which
means plus 2 hours and 30 minutes beyond utc, or -01:30:20 which means 1
hour, 30 minutes, and 20 seconds before utc (as if anyone would use
that). The following variables are generated by the system on behalf of
scripts\strut
\end{minipage}\tabularnewline
\begin{minipage}[t]{0.26\columnwidth}\raggedright\strut
\texttt{\$\$db\_error}\strut
\end{minipage} & \begin{minipage}[t]{0.10\columnwidth}\raggedright\strut
error message from a postgres failure \$\$findtext\_start - \^{}findtext
return the end normally, this is where it puts the start\strut
\end{minipage}\tabularnewline
\begin{minipage}[t]{0.26\columnwidth}\raggedright\strut
\texttt{\$\$tcpopen\_error}\strut
\end{minipage} & \begin{minipage}[t]{0.10\columnwidth}\raggedright\strut
error message from a tcpopen error\strut
\end{minipage}\tabularnewline
\begin{minipage}[t]{0.26\columnwidth}\raggedright\strut
\texttt{\$\$document}\strut
\end{minipage} & \begin{minipage}[t]{0.10\columnwidth}\raggedright\strut
name of the document being read in document mode\strut
\end{minipage}\tabularnewline
\begin{minipage}[t]{0.26\columnwidth}\raggedright\strut
\texttt{\$cs\_randindex}\strut
\end{minipage} & \begin{minipage}[t]{0.10\columnwidth}\raggedright\strut
current value of the random generator value\strut
\end{minipage}\tabularnewline
\begin{minipage}[t]{0.26\columnwidth}\raggedright\strut
\texttt{\$cs\_bot}\strut
\end{minipage} & \begin{minipage}[t]{0.10\columnwidth}\raggedright\strut
name of the bot currently in use\strut
\end{minipage}\tabularnewline
\begin{minipage}[t]{0.26\columnwidth}\raggedright\strut
\texttt{\$cs\_login}\strut
\end{minipage} & \begin{minipage}[t]{0.10\columnwidth}\raggedright\strut
login name of the user\strut
\end{minipage}\tabularnewline
\begin{minipage}[t]{0.26\columnwidth}\raggedright\strut
\texttt{\$\$csmatch\_start}\strut
\end{minipage} & \begin{minipage}[t]{0.10\columnwidth}\raggedright\strut
start of found words from \^{}match\strut
\end{minipage}\tabularnewline
\begin{minipage}[t]{0.26\columnwidth}\raggedright\strut
\texttt{\$\$csmatch\_end}\strut
\end{minipage} & \begin{minipage}[t]{0.10\columnwidth}\raggedright\strut
end of found words from \^{}match\strut
\end{minipage}\tabularnewline
\begin{minipage}[t]{0.26\columnwidth}\raggedright\strut
\texttt{\$cs\_fullfloat}\strut
\end{minipage} & \begin{minipage}[t]{0.10\columnwidth}\raggedright\strut
if defined, causes the system to generate full float 64-bit precision on
outputs, otherwise you get 2 digit precision by default\strut
\end{minipage}\tabularnewline
\begin{minipage}[t]{0.26\columnwidth}\raggedright\strut
\texttt{\$cs\_botid}\strut
\end{minipage} & \begin{minipage}[t]{0.10\columnwidth}\raggedright\strut
when non-zero creates facts and functions restricted by this bitmask so
facts and functions created by other masks cannot be seen. allows you to
separate facts and functions per bot in a multi-bot environment. During
compilation if this is set by a bot: command, then functions created and
facts created by tables will be restricted to that owner.\strut
\end{minipage}\tabularnewline
\begin{minipage}[t]{0.26\columnwidth}\raggedright\strut
\texttt{\$cs\_numbers}\strut
\end{minipage} & \begin{minipage}[t]{0.10\columnwidth}\raggedright\strut
if defined, causes the system to output numbers in a different language
style: french, indian. All other values are english.\strut
\end{minipage}\tabularnewline
\begin{minipage}[t]{0.26\columnwidth}\raggedright\strut
\texttt{\%trace\_on\ and\ \%trace\_off}\strut
\end{minipage} & \begin{minipage}[t]{0.10\columnwidth}\raggedright\strut
Pseudo system variable used by the \^{}testpattern and \^{}testoutput
call to let code request a trace be returned.\strut
\end{minipage}\tabularnewline
\begin{minipage}[t]{0.26\columnwidth}\raggedright\strut
\texttt{\$cs\_indentlevel}\strut
\end{minipage} & \begin{minipage}[t]{0.10\columnwidth}\raggedright\strut
controls indenting when tracing in \^{}testpattern. 3 is a good number
usually\strut
\end{minipage}\tabularnewline
\begin{minipage}[t]{0.26\columnwidth}\raggedright\strut
\texttt{\$indentlevel}\strut
\end{minipage} & \begin{minipage}[t]{0.10\columnwidth}\raggedright\strut
deprecated form of \$cs\_indentlevel\strut
\end{minipage}\tabularnewline
\begin{minipage}[t]{0.26\columnwidth}\raggedright\strut
\texttt{\$cs\_tracetestoutput}\strut
\end{minipage} & \begin{minipage}[t]{0.10\columnwidth}\raggedright\strut
set to 1 to force tracing in \^{}testoutput\strut
\end{minipage}\tabularnewline
\begin{minipage}[t]{0.26\columnwidth}\raggedright\strut
\texttt{\$cs\_outputlimit}\strut
\end{minipage} & \begin{minipage}[t]{0.10\columnwidth}\raggedright\strut
Generating more output than this will report a bug into
LOGS/bugs.txt\strut
\end{minipage}\tabularnewline
\begin{minipage}[t]{0.26\columnwidth}\raggedright\strut
\texttt{\$cs\_summary}\strut
\end{minipage} & \begin{minipage}[t]{0.10\columnwidth}\raggedright\strut
After volley prints to terminal milliseconds of time used in
preparation, rules, postprocessing\strut
\end{minipage}\tabularnewline
\begin{minipage}[t]{0.26\columnwidth}\raggedright\strut
\texttt{\$cs\_showtime}\strut
\end{minipage} & \begin{minipage}[t]{0.10\columnwidth}\raggedright\strut
After volley prints to terminal milliseconds of time used\strut
\end{minipage}\tabularnewline
\begin{minipage}[t]{0.26\columnwidth}\raggedright\strut
\texttt{\$cs\_new\_user}\strut
\end{minipage} & \begin{minipage}[t]{0.10\columnwidth}\raggedright\strut
set to 1, treat user as always new (don't try to read topic file)\strut
\end{minipage}\tabularnewline
\bottomrule
\end{longtable}

\section{hook functions}\label{hook-functions}

\texttt{\$cs\_beforereset} \textbar{} if set to a topic, will be
executed before :reset is executed \textbar{}\\
\texttt{\$cs\_addresponse} \textbar{} provides a function name hook onto
the output q to the user. \textbar{}\\
\texttt{\$testpatternpretopic} \textbar{} execute this topic to
preprocess input before matchines \textbar{}\\
\texttt{\$\$cs\_testpatterninput} \textbar{} a copy of user input
created by engine for \$testpatternpretopic to change if it wants
\textbar{}\\
\texttt{\$testpattern\_posttopic} \textbar{} can name a topic to be
executed after \^{}testpattern to alter returned new variables
\textbar{}

\section{variables to limit effort}\label{variables-to-limit-effort}

\texttt{\$cs\_topicretrylimit} \textbar{} if defined changes how many
times you can pass back RETRY\_TOPIC before it fails (current limit is
30) \textbar{}\\
\texttt{\$\$topic\_retry\_limit\_exceeded} \textbar{} set if topic retry
limit is encountered \textbar{}\\
\texttt{\$cs\_userhistorylimit} \textbar{} if not null, indicates how
many volleys back are tracked as what was said by both parties
\textbar{}\\
\texttt{\$cs\_sentences\_limit} \textbar{} after this many sentences in
volley, cs ignores the rest (default 50) \textbar{}\\
\texttt{\$cs\_inputlimit} \textbar{} Restrict user input size (excluding
oob) \textbar{}\\
\texttt{\$cs\_looplimit} \textbar{} loop() defaults to 1000 iterations
before stopping. You can change this default with this \textbar{}\\
\texttt{\$cs\_analyzelimit} \textbar{} in non-standalone mode, after
this millisecond limit, cs stops NL analysis of more sentences
\textbar{}\\
\texttt{\$cs\_analyzelimitlog} \textbar{} if analyzelimit triggers,
report this fact in bug log \textbar{}\\
\texttt{\$FakeTimeOffset} \textbar{} For testing analyzelimit, pretend
this much ms has already lapsed on start \textbar{}\\
\texttt{\$cs\_badspellLimit} \textbar{} x-y format. After x many
spelling corrections or x/y ratio of badspells to words seen, stop
spellchecking \textbar{}\\
\texttt{\$cs\_sequence} \textbar{} How many words in sequence to check
as a composite (default: 5) \textbar{}

\section{JSON variables}\label{json-variables}

\texttt{\$cs\_jsontimeout} \textbar{} seconds before JsonOpen declares a
time out failure. If unspecified the default is 300 \textbar{}\\
\texttt{\$cs\_saveusedJson} \textbar{} if not null, the only JSON facts
CS will write into the user's topic files that are referred to (directly
or indirectly) from user variables being saved. (see below) \textbar{}\\
\texttt{\$cs\_proxycredentials} \textbar{} See \^{}JSONOPEN in JSON
manual\textbar{}\\
\texttt{\$cs\_proxyserver} \textbar{} See \^{}JSONOPEN in JSON
manual\textbar{}\\
\texttt{\$cs\_proxymethod} \textbar{} See \^{}JSONOPEN in JSON
manual\textbar{}\\
\texttt{\$correlation\_id} \textbar{} See \^{}JSONOPEN in JSON
manual\textbar{}

\section{Mongo variables}\label{mongo-variables}

\texttt{\$cs\_mongoqueryparams} \textbar{} set as a json structure of
move its fields to a mongo query \textbar{}\\
\texttt{\$mongo\_enable\_ssl} \textbar{} if set to true, will use ssl
\textbar{}\\
\texttt{\$mongosslcafile} \textbar{} data for ssl \textbar{}\\
\texttt{\$mongosslpemfile} \textbar{} data for ssl \textbar{}\\
\texttt{\$mongosslpempwd} \textbar{} data for ssl \textbar{}\\
\texttt{\$mongovalidatessl} \textbar{} data for ssl \textbar{}\\
\texttt{\$mongo\_timeexcess} \textbar{} if certain operations exceed
this ms, log entry is created \textbar{}\\
\texttt{\$\$mongo\_error} \textbar{} error message if db not openable
\textbar{}

Note for \%trace\_on and \%trace\_off - you can use the command line
parameter \texttt{blockapitrace} to prevent tracing in any code you
accidentally leave in place.

\texttt{\$cs\_saveusedJson} exists as a kind of garbage collection.
Nowadays most facts will come from JSON data either from a website or
created in script. But keeping on top of deleting obsolete JSON may be
overlooked. When this variable is non-null, ChatScript will
automatically destroy any JSON fact that cannot trace a JSON fact path
back to some user variable. Variables that have as values the name of a
JSON object or array automatically protect all JSON facts underneath.
JSON references merely within some text string will not protect
anything, nor will references from some other non-JSON fact.

\texttt{\$cs\_inputlimit=x:y} for excessively long user input (excluding
oob portion), the input will be truncated by keeping the first x
characters and the last y characters.

\texttt{\$cs\_crash} - This topic can generate an appropriate dummy
output and CS completes that volley but does not save an updated user
file. The NEXT volley coming in will force cs to completely reload
itself before processing. Making a dummy output hopefully means the same
fatal input will not be sent back into CS to crash it again (due to
external retry when no answer is received from CS). E.g.,

\begin{verbatim}
topic: ~crashtopic system ()
    t: Huh?
\end{verbatim}

\texttt{\$cs\_addresponse} names a function of 2 arguments that will be
called when CS wants put text into the output queue of the user. The
first argument will be what CS wants to output. The second is the rule
tag that generated this output. If the function returns a failure code,
the message will be aborted and not put into the queue. If the function
returns a text value (not null) then that message will replace what was
intended to go to the user.

\end{document}
